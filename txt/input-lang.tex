\documentclass[a4paper]{article} 
\usepackage{a4}
\usepackage[T1]{fontenc}
\usepackage[latin1]{inputenc}
%\usepackage{doublespace}
\usepackage{hyperref}
%\usepackage{setspace}

\bibliographystyle{apalike}
\newcommand{\daisy}{{\sc daisy}}
\newcommand{\Daisy}{{\sc Daisy}}
\newcommand{\cplusplus}%
{{\leavevmode{\rm{\hbox{C\hskip -0.1ex\raise 0.5ex\hbox{\tiny ++}}}}}}
\newcommand{\Cplusplus}{\cplusplus}
\newcommand{\mshe}{Mike/\textsc{she}}
\newcommand{\wintel}{\texttt{win32}}
\newcommand{\dll}{\textsc{dll}}
\newcommand{\Dll}{\textsc{Dll}}
\newcommand{\gui}{\textsc{gui}}
\newcommand{\Gui}{\textsc{Gui}}
\newcommand{\unix}{Unix}
\newcommand{\dhi}{\textsc{dhi}}
\newcommand{\Dhi}{\textsc{Dhi}}
\newcommand{\api}{\textsc{api}}
\newcommand{\Api}{\textsc{Api}}
%\newcommand{\url}[1]{\linebreak[4]\texttt{<URL:#1>}}

%%% Local Variables: 
%%% mode: latex
%%% TeX-master: t
%%% End: 


\newcommand{\dname}[1]{``\texttt{#1}''}

\setcounter{secnumdepth}{0}

\begin{document}
\section*{A flexible system for specifying agricultural field management strategies}

\begin{abstract}
  Modelling the farmer is hard, a flexible system is needed.  Such a
  system is present in Daisy.  The two main abstractions in the system
  are \emph{actions} and \emph{conditions}. Actions are divided into
  field operations, waiting for conditions to be fulfilled, and
  facilities for performing multiple actions in sequence or
  simultaneously.  Conditions either directly examine the state of the
  simulation, or are Boolean operations on other conditions.  Some
  conditions examine hard-coded parts of the state directly (in
  particular \emph{time}), others allow arithmetic operations on a
  user defined subset of the state.  By combining these actions and
  conditions, quite flexible strategies can be described.  The system
  is easy to implement using modern computer science principles, and
  have been successfully used for various diverse purposes.
\end{abstract}

% \tableofcontents

\section{Modelling the farmer}

A conceptual problem for modelling the ecosystem of a field is how to
model the farmer.  The farmer is a human being, and those are
notoriously difficult to model, especially when they may have access
to the model themselves.

\paragraph{The past} When using the model on historical data, the
problem is much simplified.  The management actions have already been
performed.  The model is simply provided with a list of management
operations, at the same level of detail as the model itself.  This is
the approach traditionally taken by simulation models.  Running the
model on historical data is useful for analysing a field experiment, and
especially for extracting data that have not been directly measured.

\paragraph{The present} When using the model as part of a decision
support system (DSS), we have access to historical data, and need to
advice the farmer of what decision to take now.  The advice tends to
be hard coded heuristics on top of either a simulation model, or
purely empirical relations on the input data.  An optimization model
may find the optimal management for the field.  Optimization models
are in general very computationally intensive, and are most useful
with simple system descriptions.  For field management, an additional
obstacle for optimization models is that the weather is not known in
advance by the farmer.

\paragraph{The future} Sometimes simulations models are used for
predicting the effect of some change imposed on the field.  This
change can be to the soil (e.g.\ depletion of soil carbon
\cite{daisy-somnew}), climate (e.g. effect of global warming
\cite{climate}) or management (e.g.  introduction of irrigation).
These scenarios can also be build on historical data, with relevant
alternations.  A simulation model may be able to simulate the soil
carbon change itself.  Global warming may be imitated by simply
scaling historical data.  And irrigation may be imposed on top of an
existing management description.  The problem occurs when these
changes affect the basis for the historical management decisions.
Soil carbon depletion and irrigation are both likely to increase the
need of fertilization, and a warmer climate will almost always affect
crop phenology, and therefore harvest time.  This means the reasonable
management decisions in the hypothetical scenario will likely be
different from the historical data the scenario is based upon.  This
means that we either need to stop the scenario simulation regularly
and make the new management decision based on the new state of the
system, or use an automated decision making system.

\paragraph{Decision making systems} A sufficiently detailed DSS can be
used as a decision making system in a scenario simulation, simply by
assuming that the farmer always follow the suggestions.  The decision
making system that most faithfully models the farmer in a specific
scenario varies a lot.  Where a DSS will often try to optimize
production within legal boundaries, some scenarios (such as soil
carbon fate) will be best served by decisions that reflect common or
traditional practise.  As both the legal boundaries, common practise,
and the selection of crops varies between locations, and change over
time in the same location,, having a flexible system to describe the
strategy for making decisions is very useful.  Such a system, together
with a robust mechanistic model of the field, can also be used for
optimizing the strategy in a DSS.

In this paper we will describe a flexible system for specifying
agricultural field management strategies.  The system is implemented
as part of the Daisy agro-ecological simulation model, which we will
present first.

\section{Daisy}

Daisy \cite{daisy-fertilizer,daisy-ems} is a mechanistic simulation
model of an agro-ecological system.  It simulates water, nitrogen,
carbon and energy below and above the soil surface.  Above the soil
surface there is one or more crop, competing for light, water and
nutrients.  Plant growth is fueled by available light, and limited by
available nitrogen, water and respiration.  The accumulated growth is
divided into stem, leaf, storage organs, and roots.  For light
competition, the canopy is divided into vertical layers, where the
crops with leafs in the upper layers may shadow the crops with leafs
in the lower layer.  Below the soil surface, the soil is also divided
into vertical (for now) numerical layers.  Transport of water in the
soil matrix is done with Richard's equations, and supplemented with a
simple macropore model.  Solute transport is done with
convection-dispersion.  The user is able to choose simpler models for
transport, when computational speed is more important than accuracy.
Organic matter in the soils is divided into multiple pools, by default
6, and both short-term (days) and long-term (decades) dynamics have
been validated
\cite{willigen91-compar,vereecken91-compar,diekkruger95-compar,smith97-compar,daisy-somnew}.
The internal timestep used for most of the calculations is one hour,
but for water and solute transport this may be much shorter.

The current focus on Daisy development is on 2-dimensional transport
in the soil; macropores; pesticide and pollutant fate; as well as
partial root zone drying irrigation with hormonal signalling in the
crop; and the relationship between nitrogen content and
photosynthesis.

\section{Concepts}

In this section we will briefly explain the central concepts in the
Daisy decision making system, they will be illustrated further with
examples in the following section.  The two central concepts in our
decision making system are \emph{actions} and \emph{conditions}.  An
action has a \emph{duration} (in simulation time), and may both
examine and change the state of the simulation.  A condition is either
true or false, has no duration, and may examine but not modify the
state of the simulation.

The actions can roughly be divided into three subcategories: The first
type of actions simulate field operations, such as plowing or
harvesting.  The second type of action is to do nothing, waiting for
some condition to be true.  The third type of actions combines other
actions.  The \dname{activity} action perform a list of actions
sequentially, when the first action is finished, begin the next, and
so on.  The \dname{while} action perform a list of actions
simultaneously.  For the duration of the first action, also perform
all the remaining actions.  The \dname{if} action takes a condition
and two actions as arguments, if the condition is true, it will
perform the first action, otherwise the second.  The \dname{repeat}
action will repeat an action indefinitely.

The conditions can similarly be divided into subcategories.  The
simple conditions directly test some aspect of the simulation state,
such as whether the crop has passed a certain development stage, or
if the soil moisture at a specified depth is a above a certain
threshold.  The logic conditions allow the user to combine other
conditions with Boolean operations, such as \dname{or}, \dname{and}
and \dname{not}.  Finally, the \dname{extern} condition allows full
access to both state and arithmetic operations on the state, through
the indirection of a user specified log file.

\section{Historical management (modelling the past)}

Our first example is with historical data, where we know what the
farmer did, and when.
\begin{verbatim}
  (manager activity
    (wait (at 1987 3 20 1))
    (plowing)
    (wait (at 1987 4 4 1))
    (fertilize (N25S (weight 100.0 [kg N/ha])))
    (wait (at 1987 4 5 1))
    (sow "Spring Barley")
    (wait (at 1987 9 5 1))
    (harvest "Spring Barley"))
\end{verbatim}

First a few words about the notation.  Everything is fully
parenthesised, with the name coming after after the opening
parenthesis, followed by the arguments separated by whitespace.
Floating point numbers are followed by a dimension in square brackets,
cardinals are not.  Names containing whitespace must be surrounded
by quotation marks.

The string \dname{manager} is the name of a parameter in the
Daisy model, here we specify the value for that parameter.  The Daisy
model takes other parameters, such as \dname{column} to specify
the properties of the field, and \dname{weather}, but those are
not the topic of this article.  The next string is
\dname{activity}, which is the action that takes a list of
actions as arguments, and perform them sequentially.  In this case
there are 8 arguments, as each argument is surrounded by
parenthesizes.

The format of the action list is that we wait for a specific
condition, and then perform a field operation action
(\dname{plowing}, \dname{fertilize}, \dname{sow}, and
\dname{harvest}).  The \dname{wait} action takes a condition as
argument, in this setup the same each time.  The \dname{at}
condition takes four cardinal numbers as arguments, representing year,
month, day in month, and hour.  It is true when the simulation time
reaches that point.  The \dname{fertilize} action takes a
fertilizer as an argument, and the \dname{sow}, and
\dname{harvest} actions both takes a crop as an argument.

Not all the facilities in this example are build into Daisy.  The
\dname{N25S} fertilizer, the \dname{Spring Barley} crop, and
even the \dname{plowing} action are defined in parameterization
libraries that are distributed with Daisy in a user editable format.
The user is also able to define and name parameterizations for later
use, which leads us to our next example.

\section{Flexible crop management (modelling the future)}
    
In this example, we define a rotation consisting of two times spring
barley, followed by potatoes.
\begin{verbatim}
  (defaction sbarley activity
    (wait (mm_dd 3 20))
    (plowing)
    (wait (mm_dd 4 4))
    (fertilize (N25S (weight 100.0 [kg N/ha])))
    (wait (mm_dd 4 5))
    (sow "Spring Barley")
    (wait (mm_dd 9 5))
    (harvest "Spring Barley"))
  
  (defaction barbarpot activity
    sbarley sbarley potato)
  
  (manager barbarpot)
\end{verbatim}
The first thing we notice is the \dname{defaction} keyword.  This is
an abbreviation for \emph{define action}, and simply associate its
first argument, a name, with the following arguments, an action.  We
can then later refer to the defined action by the name used.  We do
that when we define the \dname{barbarpot} action, which again is used
as the value for the \dname{manager} parameter of the Daisy simulation
model.  The \dname{sbarley} action consist of the same sequence of 8
actions as in the first example, with a single difference.  Instead of
specifying the absolute simulation time in the \dname{wait} actions,
we only specify month and day in the month.  This \dname{mm\_dd} will
be true once every year (at 8AM by default).  This allows us to use
the \dname{sbarley} action repeatedly in a rotation, here together
with a similar \dname{potato} action not shown here.

\subsection{Simple state}

The problem is that due to climate variation, we are unlikely to do
perform the same actions at the same date for different years.  So we
modify the example to depend on other state than just the simulation
time. 
\begin{verbatim}
  (defcondition trafficable 
    (and (not (soil_water_pressure_above (height -10.0 [cm])
                                         (potential -50.0 [cm])))
         (soil_temperature_above (height -10.0 [cm]) 
                                 (temperature 0.0 [dg C]))))
  
  (defaction sbarley activity
    (wait (mm_dd 3 20))
    (wait (or trafficable (mm_dd 4 3)))
    (plowing)
    (wait (mm_dd 4 4))
    (fertilize (N25S (weight 100.0 [kg N/ha])))
    (wait (mm_dd 4 5))
    (sow "Spring Barley")
    (wait (or (crop_ds_after "Spring Barley" 2.0)
              (mm_dd 10 1)))
    (harvest "Spring Barley"))
\end{verbatim}
Here, we first define a \dname{trafficable} condition.  The idea
behind this is that we want to test if the simulated soil is in a good
condition for use with farm vehicles.  The condition is of type
\dname{and}, which is true if and only if all the arguments are true.
Here, the first test is whether the soil is too wet.  We test the
state 10 cm down the soil, and see if the potential is above -50 cm
water pressure.  If this is \emph{not} true, the soil is sufficiently
dry to carry traffic.  At the same time, we don't want to do tillage
operations on frozen soil, so the second condition is that the
temperature at 10 cm depth must be above 0 $^\circ$C.  We use the
\dname{trafficable} condition before plowing, but inside an \dname{or}
condition to ensure that plowing happens before sowing. For harvest,
we use the built-in \dname{crop\_ds\_after} condition to ensure that
the spring barley is ripe, which is at development stage 2.0 in the
Daisy model.

With these modifications, we have a definition of spring barley
management that can be used in a rotation with variable weather.

\subsection{Concurrency}

Until now, all our management examples have been build upon the
\dname{activity} action, performing one task at a time.  We could
do the same with irrigation, but defining it separately can simplify
it, and illustrates how we provide concurrency.
\begin{verbatim}
  (defaction irrigate_30 activity 
    (wait (and (after_mm_dd 6 1)
               (before_mm_dd 8 15)
               (not (soil_water_pressure_above 
                      (height -25.0 [cm])
                      (potential -1000 [pF])))))
    (irrigate_overhead 5 [mm/h] (hours 6))
    (wait_days 2))
\end{verbatim}
Here we first define a single irrigation activity called
\dname{irrigate\_30}.  We wait until the soil is sufficiently dry
(more than 3 pF) at 25 cm.  We also check that we are within the local
irrigation season with the \dname{after\_mm\_dd} and
\dname{before\_mm\_dd} conditions\footnote{These test the date
  relative to the calendar year, and are most useful in Northern
  climates where it is too cold to grow crops in December/January.}.
When all three conditions are true, we irrigate with 5 mm/h for 6
hours.  Afterwards, we wait two days in order to ensure the water
reach our measuring point at 25 cm.
\begin{verbatim}
  (defaction irr_sbarley
    (while sbarley
      (repeat irrigate_30)))
\end{verbatim}
Our irrigated spring barley now demonstrates two actions.  The first
is \dname{while}, which takes a list of actions as arguments, in this
case two.  The first is our already defined \dname{sbarley} action,
which defines a (non-irrigated) management for a spring barley.  This
action controls the duration of the \dname{while} action, so our
irrigated barley will have the same duration as out non-irrigated
barley.  For that duration the second argument is also active.  The
second argument is a \dname{repeat} action.  The \dname{repeat} action
takes a single argument, and repeat it indefinitely.  Here the
duration is limited because the repeat is used inside a \dname{while}, and
the duration of the \dname{while} is limited by the first
argument.

\section{Advanced state (modelling the present)} 

There are not build-in conditions for all the state in Daisy, only the
part most commonly used.  And the only thing the conditions can do
with them, is to compare them with constants.  However, all parts of
the state of Daisy can be written to log files during the simulation.
We have used this to create an even more flexible system, especially
suitable for DSS use.

\subsection{The \dname{extern} log}

First we must define what part of the state we want to examine.
\begin{verbatim}
  (deflog DSS extern
    (when hourly)
    (entries (interval (tag "NO3-N 0-30 cm")
                       (from 0 [cm]) (to   -30 [cm])
                       (path column "*" SoilNO3 M)
                       (spec fixed SoilNO3 M)
                       (dimension [kg N/ha]))
             (interval (tag "NH4-N 0-30 cm")
                       (from 0 [cm]) (to   -30 [cm])
                       (path column "*" SoilNH4 M)
                       (spec fixed SoilNH4 M)
                       (dimension [kg N/ha]))
             (number (path column "*" Vegetation crops crops "*"
                           Devel "*" DS)
                     (spec phenology default DS))
             (number (path column "*" Vegetation crops crops "*"
                           CrpN PtNCnt)
                     (spec fixed CrpN PtNCnt)
                     (dimension "kg N/ha"))
             (number (path column "*" Vegetation crops crops "*"
                           CrpN CrNCnt)
                     (spec fixed CrpN CrNCnt)
                     (dimension "kg N/ha"))
             (number (tag AcNCnt)
                     (path column "*" Vegetation crops crops "*"
                           Prod NCrop)
                     (spec fixed Production NCrop)
                     (dimension "kg N/ha"))))
\end{verbatim}
Here we define a log named \dname{DSS} of the type \dname{extern}.  A
similar system is used by Daisy for defining what part of the state to
write to files during the simulation, the only differences are that 1)
the name of the log type is \dname{table} instead of \dname{extern},
and 2) an additional parameter \dname{where} specifies the name of
file to store the result.  For both \dname{table} and \dname{extern}
logs, the \dname{when} parameter specifies how often to store the
results, and the \dname{entries} parameter specifies which part of the
state to store.

\paragraph{Select methods} The \dname{entries} parameter takes a list
of select methods for individual state variables within the system.
Here we use two types of select methods, \dname{interval} and
\dname{number}.  Both have some shared parameters: \dname{path} and
\dname{spec} specify how to find the state variable within the
internal hierarchical structure of Daisy\footnote{Described in
  \cite{daisy-ref}, and outside the scope of this article}, and
\dname{dimension} the desired dimension for the value\footnote{Daisy
  will convert from the internal dimension if needed.}.  Finally,
\dname{tag} is the external name (the use of which we will come back
to) for the value, if not specified the last element of the
\dname{path} list is used.  The \dname{interval} select method is used
for soil state only, and will integrate the soil content from the
depth specified by the \dname{from} and \dname{to} parameters.  The
values of these two parameters are height above soil surface, so 0 cm
is the soil surface, and -30 cm is 30 cm below the soil surface.

\subsection{The \dname{extern} condition}

We can access the information in \dname{extern} logs from the
management system using the similarly named \dname{extern} condition.
\begin{verbatim}
  (defcondition DSS_fert
    (extern DSS 
      (and (< 0 [] DS 1.8 [])
           (< (+ "NO3-N 0-30 cm" "NH4-N 0-30 cm") 15 [kg N/ha])
           (or (> (- PtNCnt AcNCnt) 20 [kg N/ha])
               (> (- PtNCnt AcNCnt) (- AcNCnt CrNCnt))))))
\end{verbatim}
The first argument to the condition is the name of the \dname{extern}
log that defines the state information we examine.  The second
argument is a \emph{boolean}. 

\paragraph{Booleans and numbers} A boolean is similar to a condition
in that it is either true or false, and indeed, the value of the
\dname{extern} condition will be the value of its second argument.
Booleans, as one might guess from the name, also provides the usual
operations on Boolsk algebra.  Additionally, booleans provides
comparison operations ($<$ and $>$) on \emph{numbers}, another new
abstraction here.  A number can be either a constant (like \texttt{1.8
  []} in the example above, number constants have dimensions), an
arithmetic expression (\texttt{-} and \texttt{+} are used in the
example, but a full range of arithmetic operations and functions are
available), or finally the name of one of the entries in the log
specified by the first argument of the \dname{extern} condition.  The
name of the entry is either the value of the \dname{tag} parameter if
specified, or the last entry of the \dname{path} parameter.  When the
number is evaluated, the latest value for that entry in the
\dname{extern} log will be used.

\paragraph{Explanation} In the example above, we let fertilization
depend on both crop phenology, available nitrogen in the soil, and the
nitrogen concentration of the crop.  \dname{DS} refers to the
developing stage of the crop.  We here require it to be between 0 and
1.8, when translates to emergence, and nearly ripe. We also require
that the sum of nitrate-nitrogen and ammonium-nitrogen in the top 30
cm should be less than 15 kg/ha.  And finally we we look at the
nitrogen status in the crop.  Here, \dname{AcNCnt} refers to the
actual nitrogen content of the crop.  \dname{PtNCnt} refers to the
potential nitrogen content of the crop.  And finally, \dname{CrNCnt}
refers to the critical nitrogen content of the crop.  If the actual
content falls below critical, dry matter production is affected.
Everything above critical is luxury uptake.  \dname{PtNCnt} and
\dname{CrNCnt} depend on the age and dry matter content of the crop.
Our requirement for fertilizing is now that the crop either has room
for an additional 20 kg N/ha before reaching its potential, or that
the crop is closer to the critical than the potential.

\subsection{Putting it all together}

This condition can now be used in a fertilizer strategy, similar to
how we earlier defined an irrigation strategy.
\begin{verbatim}
  (defaction fertilize_DSS activity 
    (wait DSS_fert)
    (fertilize (N25S (weight 20.0 [kg N/ha])))

  (defaction fert_sbarley
    (while sbarley
      (repeat fertilize_DSS)))
\end{verbatim}
Or we could combine them.
\begin{verbatim}
  (defaction fert_irr_sbarley
    (while sbarley
      (repeat irrigate_30)
      (repeat fertilize_DSS)))
\end{verbatim}

\section{Implementation}

Since all expressions are fully parenthesised, the input language can
be implemented with a trivial recursive-descend parser with a single
token lookahead, and no backtracking.  Alternatively, if one is
willing to accept a somewhat bulkier notation, a ready-to-use XML
based parser could be used.

The base concepts \emph{action}, \emph{condition}, \emph{boolean}, and
\emph{number} are all represented by pure base class, and the specific
actions, conditions, etc.\ by concrete classes derived from the base.
The accessor functions for actions and conditions need to take the
simulation state as an argument (for conditions it can be a read-only
argument).

Each concrete class is responsible for registering itself with name,
constructor and parameters with a factory pattern.  This registry is
also use by the parser.  A global table of extern logs is used for
making the log information available in the numerical expressions
within the ``extern'' condition.

In \cite{daisy-ems} more information about the implementation
framework can be found.

\section{History}

The original Daisy implementation in \textsc{Fortran} from 1989
provided for only historical management, with a few hardcoded
exceptions such as an early version of \dname{trafficable} condition,
to delay the specified tillage operation dates.  In 1995 the Danish
Informatics Network in Agriculture (Dina) funded a rewrite in
\cplusplus{} based on the computer science principles mentioned in the
previous section.  This rewrite laid the foundation for the current
system.  Originally a simple rule based system was implemented, with a
list of condition and action pairs.  If the condition was true, the
action was performed.  This system is still available as a special
action named \dname{cond}, but proved unsuitable for teaching
purposes.  So it was supplemented with the \dname{activity} action,
which required actions to have a duration.  The \dname{while} and
\dname{repeat} action were implemented shortly after, in order to
describe irrigation strategies independently of the specific crop on the
field.

The \dname{extern} log model was implemented in 1999, as a part of a
flexible interface to the Mike/SHE groundwater model \cite{mikeshe}
under the the SDSS EU project \cite{sdss}.  The ``number'' abstraction
for doing arithmetic was implemented in 2004 to support generic
pedotransfer functions for sorbtion in a Ph.D. project about dissolved
organic matter \cite{bgj-dom}.  The last components were provided as
part of the FertOrgaNic project \cite{fertorganic}, where we should
develop DSS rules in a project with 3 experimental years and 9
experimental sites across Europe.  In 2005 the ``boolean'' abstraction
was implemented and the ``number'' abstraction significantly extended
as part of a subsystem that should plot the large amountof field data
together with simulation results.  Finally, in 2006 the \dname{extern}
condition and implemented and the ``boolean'' abstraction further
developed in order to implement the DSS directly in Daisy.

\section{Applications}

The FertOrgaNic project is the first use of the system to implement a
DSS, which is where the advanced state management comes to its right.
A running demo can be found on the project home page,
\url{http://www.fertorganic.org/}.  We hope many more DSS uses will
follow, especially as the advanced state management system was only
developed near the end of the project.

The historical data management is used in basically every project
where we have experimental data we want to analyse with the model,
including the FertOrgaNic project.

\paragraph{Flexible crop management use by the adminsitration} The
flexible crop management system has been used mainly by the Danish
environmental administration.  One example is a standardized procedure
\cite{daisy-staabi} for getting an environmental permit for major
changes in farm management, such as introducing pigs to what used to
be a pure crop farm.  The procedure calls for running a crop rotation
with the new and old farm management with 10 years of historical
weather data, and compare the nitrogen leaching predicted by Daisy.
This simulation must be repeated several times with the weather data
shifted one year, so each crop in the rotation is grown under each
year in the weather data.  Setting up management for all these
combinations by hand would be unmanageable and errorprone, so the
flexible crop management system, where we only have to describe each
crop once, has been a great help.

Another example is the program for evaluating the effect of the
national ``Action Plan for the Aquatic Environment''.  In selected
areas the aquatic environment is intensively monitored (the
\textsc{Loop} areas), and this information is used together with
statistical information about the farm management and soil types in
the area to model the nitrogen leaching \cite{novana}.  Since the
regional authorities don't have the exact information, they instead
use statistical combinations of soil types and typical crop management
to estimate the total leaching.  This, again, is only possible because
the crop management can be described independently of the individual
farm.

\paragraph{Flexible crop management use in research} In a scientific
context, the crop management is often used for testing hypothetical
scenarios.  This is also useful for educational purpose, asking
students to introduce irrigation to a simulated and see the effect in
nitrogen is an example of a typical exercise using the flexible crop
management system.  An example of a scientific use is found in
\cite{boegh04a}, where a water balance for Zealand is calculated based
on statistical information in a GIS setup, another in \cite{org-init},
where we examine how best to initialise the organic matter pool based
on 1000 year simulations where we go from equilibrium under one
management strategy, to equilibrium under another management strategy.
In the first example we scale up in space, in the second we scale up
in time.

\section{Usability}

We have taught the system to agronomy students, and other academic
students, as well as to administrators with a background in science,
and other scientist.  In our experience, people with a programming
background find it intuitive, people with no programming knowledge
less so.  However, when we start with the system for historical data
management and build from there, they will in general understand it.
Most people who later use the system for real work learn to appreciate
the flexibility.

There exist several graphical user interfaces for Daisy
\cite{plantinfo-daisy,daisygis,fertorganic}, which generate management
descriptions using a subset of the textual interface described in this
article.  Most users of these interfaces tend to run into the limits
eventually, and switch to the textual interface.  However, for users
not accustomed to textual program interfaces, the graphical user
interfaces may seem less intimidating.

\section{Conclusion}

The ``action'' and ``condition'' abstractions we choose in the 1995
rewrite have proved very flexible, and provided the basis for several
different systems for describing the management component of the
model.  The flexible crop management system have proved useful in the
administration, for teaching, and for research.  The latest system, the
advanced state, has not yet proved itself, but looks promising.  

All of these are easily implementable using
elementary\footnote{Elementary today, they were not all elementary in
  1995.}  computer science tools, and we believe that the diverse
applications of the system demonstrate the value of combining the the
specific modelling expertise behind the original Daisy model, with the
computer science expertise the 1995 rewrite added.

\section{Acknowledgements}

This work was financed by a large array of projects, most of which had
their focus elsewhere.  We do believe that the Danish Informatics
Network in Agriculture (Dina), which funded the rewrite that made it
all possible, deserves special mention, together with FertOrgaNic
(Improved organic fertiliser management for high nitrogen and water
use efficiency and reduced pollution in crop systems), EU 5th
Framework RTD project (QLK5-2002-01799), which financed the advanced
state management.

\addcontentsline{toc}{section}{\numberline{}Bibliography}
\bibliography{daisy}

\end{document}
