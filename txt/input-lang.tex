\documentclass[a4paper]{article}
\usepackage{a4}
\usepackage[T1]{fontenc}
\usepackage[latin1]{inputenc}
%\usepackage{doublespace}
%\usepackage{hyperref}

\bibliographystyle{apalike}
\newcommand{\daisy}{{\sc daisy}}
\newcommand{\Daisy}{{\sc Daisy}}
\newcommand{\cplusplus}%
{{\leavevmode{\rm{\hbox{C\hskip -0.1ex\raise 0.5ex\hbox{\tiny ++}}}}}}
\newcommand{\Cplusplus}{\cplusplus}
\newcommand{\mshe}{Mike/\textsc{she}}
\newcommand{\wintel}{\texttt{win32}}
\newcommand{\dll}{\textsc{dll}}
\newcommand{\Dll}{\textsc{Dll}}
\newcommand{\gui}{\textsc{gui}}
\newcommand{\Gui}{\textsc{Gui}}
\newcommand{\unix}{Unix}
\newcommand{\dhi}{\textsc{dhi}}
\newcommand{\Dhi}{\textsc{Dhi}}
\newcommand{\api}{\textsc{api}}
\newcommand{\Api}{\textsc{Api}}
%\newcommand{\url}[1]{\linebreak[4]\texttt{<URL:#1>}}

%%% Local Variables: 
%%% mode: latex
%%% TeX-master: t
%%% End: 


\begin{document}

\section*{A flexible system for specifying agricultural field management strategies}

\begin{abstract}
  Modelling the farmer is hard, a flexible system is needed.  Such a
  system is present in Daisy.  The two main abstractions in the system
  is actions, and conditions. Actions are divided into field
  operations, waiting for conditions, and two special facilities for
  performing multiple actions in sequence, or at the same time.
  Conditions either directly examine the state of the simulation, or
  are Boolean operations on other conditions.  Some conditions examine
  hard-coded parts of the state directly (in particular time), others
  allows allow arithmetic operations on a user defined subset of the
  state.  By combining these actions and conditions, quite flexible
  starategies can be described.
\end{abstract}

\tableofcontents

\section{Modelling the farmer}

A conceptual problem for modelling the ecosystem of a field is how to
model the farmer.  The farmer is a human being, and those are
notoriously difficult to model, especially when they may have access
to the model themselves[REF].

\paragraph{The past} When using the model on historic data, the
problem is much simplified.  The management actions have already been
performed.  The model is simply feed with a list of mangement
operations, at the same level of detail as the model itself.  This is
the approach traditionally taken by simulation models[REF].  Running
the model on historic data is useful when analysing a field
experiment, and when trying to extract data that have not been
directly measured, such as nitrogen leaching.

\paragraph{The present} When using the model as part of a decision
support system (DSS), we have access to historic data, and need to
advice the farmer of what decision to take now.  We thus also here
avoid the need to model the farmer.  The advice tends to be hard coded
heuristics on top of either a simulation model [REF] or purely
empirical relations on the input data [REF].  An optimization model
may find the optimal management for the field.  Optimization models
are very computationally intensive, and are most usefyl for simple
system descriptions.  For field management, an additional obstacle for
optimization models is that the weather is not known in advance by the
farmer.

\paragraph{The future} Sometimes simulations models are used for
predicting the effect of some change imposed on the field.  This
change can be to the soil (eg.\ depletion of soil carbon[REF]),
climate (e.g. effect of global warming[REF]) or management (e.g.
introduction of irrigation[REF]).  These scenarios can also be build
on historical data, with proper alternation.  A simulation model may
be able to simulate the soil carbon change itself.  Global warming may
be immitated by simply scaling historical data.  And irrigation may be
imposed on top of an exsisting management discription.  The problem
occurs when these changes affect the basis for the historical
management decisions.  Soil carbon depletion and irrigation are both
likely to increase the need of fertilization, and a warmer climate
will almost always affect crop phenology, and therefore harvest time.
This means the sensible management decisions in the hypothetical
scenario will likely be different from the historical data the
scenario is based upon.  This means that we either need to stop the
scenario simulation regularly and make the new managment decision
based on the new state of the system, or use an automated decision
making system, i.e.\ a model of the farmer.

\paragraph{Decision making systems} A sufficiently detailed DSS can be
used as a decision making system in a scanario simulation, simply by
assuming that the farmer always follow the suggestions.  Scenario
simulations can also be used for developing the DSS, by examining what
would happen if the farmer followed the suggestions.  The decision
making system that most faithfully models the farmer in a specific
scenario varies a lot.  Where a DSS will often try to optimize
production within legal boundaries, some scenarios (such a soil carbon
fate) will be best served by decisions that reflect common or
traditional practise.  As both the legal boundaries, common practise,
and the selection of crops varies between locations, and change over
time in the same location,, having a flexible system to describe the
strategy for making decisions is very useful.  Such a system, together
with a robust mechanistic model of the field, can also be used for
optimizing the strategy in a DSS.

In this paper we will describe a flexible system for specifying
agricultural field management strategies.  The system is implemented
as part of the Daisy agro-ecological simulation model, which we will
describe first.

\section{Daisy}

Daisy\cite{daisy-fertilizer,daisy-ems} is a mechanistic simulation
model of an agro-ecological system.  It simulates water, nitrogen,
carbon and energy below and above the soil surface.  Above the soil
surface there is one or more crop, competing for light, water and
nutrients.  Plant growth is fueled by available light, and limited by
available nitrogen, water and respiration.  The accumulated growth is
divided into stem, leaf, storage organs, and roots.  For light
competition, the canopy is divided into vertical layers, where the
crops with leafs in the uppper layers may shadow the crops with leafs
in the lower layer.  Below the soil surface, the soil is also divided
into vertical (for now) numerical layers.  Transport of water in the
soil is done with Richard's equations for the matrix, suplemented with
a simple macropore model.  Solute transport is done with
convection-dispersion (and macropores).  The user is able to choose
simpler models for transport, when computational speed is more
important than accuracy.  Organic matter in the soils divided into
multiple pools, by default 6, and both short-term (days) and long-term
(decades) dynamics can be simulated.  Daisy has been validated under a
number of occations [REF], sometimes comming up as best of breed in
comparitative studies. 


The current focus on Daisy development is on 2-dimensional transport
in the soil, macropores, pesticide and pollutant fate, as well as
partial root zone drying irrigation with hormonal signalling in the
crop, and the relationship between nitrogen contant and
photosynthesis.


\section{Concepts}

actions
  sequential and concurrent
  farm operations and waiting
conditions

named parameterizations

\section{Historical management}

\begin{verbatim}
  (manager activity
    (wait (at (2006 10 1 1)))
    (plowing)
    ...)
\end{verbatim}

\section{Crop management}
    
\begin{verbatim}
  (defaction sbarley activity
    (wait (mm_dd 10 1))
    (plowing)
    ...)

  (defaction rotation activity
    sbarley sbarley potato)

  (manager rotation)
\end{verbatim}
   
\section{Simple state}


\begin{verbatim}
  (defcondition trafficable ...)

  (defaction sbarley activity
    (wait (mm_dd 10 1))
    (wait (or trafficable (mm_dd 11 1)))
    (plowing)
    ....
    (wait (or (crop_DS_above all 2.0) (mm_dd 9 1)))
    (harvest "Spring Barley"))
\end{verbatim}

Defcond.  Fjernede �ret.  Moden.  Logic.

\section{Concurrency}

\begin{verbatim}
  (defaction irrigation ...)

  (defaction irrigated_sbarley activity
    (while sbarley
       irrigation))
\end{verbatim}

\section{Advanced state}

\begin{verbatim}
  (deflog DSS extern ...)

  (defnumber N_min ...)

  (defaction sbarley ... (extern DSS))
\end{verbatim}

\section{Implementation}

base class

factory [REF]

\section{Conclusion}

Vi er s� gode.

\section{Acknowledgements}

DINA, FertOrgaNic.

\addcontentsline{toc}{section}{\numberline{}Bibliography}
\bibliography{daisy}

\end{document}
