%%\documentclass[5p,authoryear]{elsarticle}
\documentclass[1p,authoryear,review]{elsarticle}
\usepackage{mathptmx}

\usepackage[T1]{fontenc}
\usepackage[latin1]{inputenc}
\usepackage{amsmath}
\usepackage{hyperref}
\usepackage{upgreek}
\usepackage[misc]{ifsym}
\usepackage[dvips]{color}
\usepackage[left,running]{lineno}

\newcommand{\twod}{\textsc{2d}}
\newcommand{\oned}{\textsc{1d}}
\newcommand{\threed}{\textsc{3d}}
 \newcommand{\cplusplus}%
{{\leavevmode{\rm{\hbox{C\hskip -0.1ex\raise 0.5ex\hbox{\tiny ++}}}}}}

\newcommand{\MyEq}[1]{Eq.~\ref{#1}}


\begin{document}

\linenumbers

\title{A simple parametric \twod{} root density distribution model
  for row crops}  

\author[ku]{Per Abrahamsen\corref{cor}}
\ead{pa@life.ku.dk}
\author[au]{Mathias Neumann Andersen}
\author[rd]{Mikkel Mollerup}
\author[caas]{Xuebin Qi}
\author[caas]{Dongmei Qiao}
\author[caas]{Ping Li}
\author[cau]{Guitong Li}
\author[ku]{S�ren Hansen}

\cortext[cor]{Corresponding author}
\address[ku]{Department of Basic Sciences and Environment, 
  Faculty of Life Sciences, University of Copenhagen, 
  Thorvaldsensvej 40, DK-1871, Frederiksberg, Denmark}
\address[au]{Department of Agroecology, University of Aarhus,
  Blichers All� 20, DK-8830, Tjele, Denmark}
\address[rd]{Road Directorate, Ministry of Transport, Denmark}
\address[caas]{Department of Water Resources and Water Environment,
  Farmland Irrigation Research Institute,
  Chinese Academy of Agricultural Science,
  380 East Hongli Road, Xinxiang City, Henan Province, 453003, P.R.China}
\address[au]{Department of Soil and Water Science,
  College of Resources and Environment, China Agricultural University,
  No.2 of West road of Yuanmingyuan, Haidian, Beijing, 100193, China}

\begin{abstract}
  Water saving techniques such as alternating partial root zone
  techniques can be evaluated by mechanistic modelling of soil water
  and plant growth. This requires knowledge of the root density
  distribution, which is expensive to measure. The objective of this
  paper is to present a simple and robust model of root density
  distribution for row crops, based on physical quantities that can be
  estimated by a plant growth model.
  
  An empirical model of vertical root density distribution is extended
  to two dimensions, and the parameters are derived in terms of root
  mass and root zone width and depth. Root density for potatoes and
  tomatoes were measured at two sites.  Both the original and the
  extended model are fitted to these datasets. An F-test is used to
  determine is the extended model is significantly better than the
  original model.  The extended model is superior in the cases where
  the root zone width is less than the distance between rows.  The
  median of the coefficient of determination (R$^2$) for all the
  examined datasets was 0.86, and a visual inspection confirms that
  the model accounts for both the horizontal and vertical root density
  variation.  Our simple model thus were able to adequately describe
  the measured data, relying on three quantities, all physically
  meaningful.
\end{abstract}

\begin{keyword}
  Root length density \sep Root modeling \sep Row crops 
  \sep Partial root zone drying
\end{keyword}

\maketitle

\section{Introduction}

With alternate side partial root zone drying (PRD) irrigation,
transpiration will depend on the spatial distribution of the roots and
water in the soil, due to root signals. Several root models
suitable for PRD exists. \@~\cite{vrugt2001calibration} e.g.\ proposes
a flexible, empirical \twod{} root model with a radial geometry, which
\cite{apri} successfully used to model PRD irrigation of a vineyard.
A static description of the root density based on measurements was
used in the latter application.  For perennial plants, using a static
description of the root system during the irrigation season can make
sense, but for annual plants, such as most vegetables, the root zone
will vary extensively over the season.  A dynamic root growth model
for row crops based on diffusion theory was proposed
in~\cite{acock1996convective}. In general this model required a
numerical solution, but analytical solutions for special cases were
found by~\cite{willigen2002two}.

The \twod{} radial and the \threed{} root density models typically
include only a single plant.  In contrast, \oned{} models by their
nature are concerned with a whole population, like a forest or a field
crop.  An example of the latter would be the model of~\cite{gp74}
based on a meta-analysis, where they found an exponential decrease
with depth to capture the field observations.  These root models 
are parametric, with the roots described as a density function $L\,
(x,z)$, where the independent variables are horizontal distance to the
plant $x$ and vertical distance to the soil surface $z$.  In contrast,
architectural root models track the structure and branches of the
roots.  \cite{wang2004modelling} gives an overview of existing
\oned{}, \twod{}, and \threed{} root models, both parametric and
architectural.

Radial coordinates are a natural choice for looking at the root zone
of a single plant.  But when the root zones of the plants overlap, the
problem becomes \threed{}.  Solving Richards' equation in \threed{} is
complex and results in long simulation times.  Finding a \twod{}
approximation is therefore desirable.  

\citet{pedersen2010modelling} suggested a simple \twod{} extension of
the model by \cite{gp74} for row crops, where the root density
decreased exponentially with the horizontal distance to the row.  The
assumption is that the roots are sufficiently dense in the direction
of the rows for us to ignore that dimension.  Like \cite{gp74}, it is
a static, empirical model. This model was validated indirectly, through
the use of soil water measurements. In the present paper we suggest a
similar model, with the following improvements: 1) The new model will
handle overlapping root zones from multiple rows; 2) The model
parameters will be calculated from physical rather than empirical
quantities; 3) The model will be tested on two independent sets of
directly measured root densities.  The proposed model is static in
that it doesn't include root growth, but it can be parametrized
dynamically based on root mass and the depth and width of the root
zone when combined with a suitable crop growth model.

The new root model was used in the SAFIR (SAFe IRrigation) EU project
\citep{safir2010} as part of the Daisy agro-ecological model
\citep{daisy-fertilizer,daisy2012} for calculation of water and solute
uptake.  The SAFIR project had two aspects, the first being assessment
of the risk for human health with regard to use of wastewater for
irrigation \citep{Forslund2010440,Surdyk2010451}, and the second was
water saving irrigation techniques, in particular PRD
\citep{Jensen2010403}.  There were five experimental sites, three in
Europe, and two in China, and three crops, potatoes, fresh tomatoes,
and processing tomatoes.  For the two Chinese sites root density was
measured. The original Daisy soil water model was extended from
\oned{} to \twod{}, and used for modelling the experiments.  The
original Daisy crop model \citep{eupotato} calculates root dry mass
and root zone depth, and assumes a vertical root density distribution
following the model by \cite{gp74}.  With the \twod{} extension, root
zone width was added to the crop model, as well as the root density
distribution model discussed in this paper.

Daisy facilitates replaceable descriptions of the physical processes in
the system, as described in \citet{daisy-ems}. The main relevant
processes of the specific setup used in SAFIR is depicted on
Fig.~\ref{fig:daisy}. The crop growth and uptake modules are described
in \citet{daisyN}, and the SVAT model in \citet{ssoc}. The new \twod{}
root density model, the topic of this paper, has not been presented in
any of the previous papers, nor has the root density measurement from
the two Chinese sites been published before. 

\begin{figure}[htbp]
  \begin{center}
    \input{daisy-rootdens.latex}
  \end{center}
  \caption{The root density model (red) in the context of Daisy as
    applied in the SAFIR project. Boxes represent modules, arrows
    represent information flow. Only a subset of both have been
    included. The crop growth module delivers the total root mass
    ($M_r$), root zone depth ($d_c$), and root zone width ($w_c$) to
    the root density module, which delivers a spatial distribution of
    root density ($L$) to the uptake module. The uptake module is
    driven by potential transpiration and N demand, which come from
    the Soil-Vegetation-Atmosphere transfer (SVAT) module and the crop
    growth module, respectively.  Using the root density, the actual
    uptake is then calculated the hydraulic conductivity ($K$), the
    pressure head ($h$), and the nitrogen solute concentration ($C_N$)
    of nitrate and ammonium.  The later three are delivered by the
    soil module, which receive sink terms for water and nitrogen
    ($S_\Theta$ and $S_N$) in return. The crop growth is mainly driven
    by assimilated CO$_2$, which is calculated by the SVAT module,
    based on weather data, actual transpiration, and the leaf area
    (LAI) and nitrogen status of leafs (Leaf-N).}
  \label{fig:daisy}
\end{figure}

\section{Materials and Methods}

\subsection{Model theory}

Our goal is to describe the two dimensional root density distribution
for row crops based on root dry mass, root zone depth, and root zone
width.  These values can be provided by a general crop growth model
(like the one found in Daisy), and furthermore have the advantage that
they all have a physical meaning.  However, first we will describe a
traditional one dimensional root density distribution in terms of root
dry mass and root zone depth.  This will provide a reference for later
comparison.

\subsubsection{Densely populated fields}
\label{sec:model-1d}

In accordance with \cite{gp74}, the root density distribution $L\,(z)$
for a crop can be described by
\begin{equation}
  L\,(z) = L_0\, e^{-a z}
  \label{eq:g+p}
\end{equation}
where $L_0$ is the root density at the soil surface, $a$ is a
distribution parameter, and $z$ is the depth below soil surface.  We
here assume that the density is uniformly distributed on the
horizontal plane, an assumption that fails with e.g.\ row crops.  The
parameters $a$ and $L_0$ will both vary with time.  For a production
oriented simulation model like Daisy, it can be more convenient to
specify the density in terms of accumulated root dry matter $M_r$ and
total root zone depth $d_c$ \citep{daisyN}.

We define the root zone depth at the depth where the root density is at a
specified threshold $L_m$. The specific value chosen for $L_m$ would
depend on the model that is going to to provide $d_c$. By inserting
this relation in \MyEq{eq:g+p}, we get
\begin{equation}
  L_m = L_0\, e^{-a \, d_c}
  \label{eq:root-depth}
\end{equation}
We convert the root mass to root length $l_r$ by assuming the specific
root length $S_r$ is a known constant
\begin{equation}
  l_r = S_r \, M_r
  \label{eq:root-length}
\end{equation}
The total root length is also the integral of the root density over
the profile
\begin{equation}
  l_r = \int_0^{\infty} L\, (z) \: dz 
      = \int_0^{\infty} L_0\, e^{-a z} \, dz 
      = \frac{L_0}{a}
  \label{eq:root-integral}
\end{equation}
By inserting the expression we get for $L_0$ from
\MyEq{eq:root-integral} in \MyEq{eq:root-depth} we get
\begin{equation}
  L_m = l_r \, a \, e^{-a \, d_c}
  \label{eq:a-only}
\end{equation}
If we substitute $W = -a \, d_c$ and isolate the known values on the right
side this gives us
\begin{equation}
   W e^W = - L_m \frac{d_c}{l_r}
  \label{eq:Lambert}
\end{equation}
If the root length is suffiently high the equation will have multiple
solutions, the solution for $W\le-1$ corresponds to a high density
near the surface that decreases sharply with depth, and is more
realistic than the solution for $W>-1$ which has a comparable flat
root profile, where more of the roots lie outside the root zone.  The
solution can be found using Newton's method and an initial quess of
$-2$.  After finding a value for $W$, we can find the desired density
parameters $L_0$ and $a$ by substituting back
\begin{eqnarray}
   a   &=& -W / d_c\\\label{eq:a-solved}
   L_0 &=& \frac{L_m}{e^{-a \, d_c}} = L_m e^{a \, d_c}\label{eq:L0-found}
\end{eqnarray}

\subsubsection{Row crops}
\label{sec:model-2d}

We can describe a row crop with a two dimensional model by assuming
that the roots are uniformly distributed along the direction of the
row.  Our second dimension $x$ is horizontal, orthogonal to the row.
The root density at a specific point can be denoted $L\, (x,z)$, and
we choose a coordinate system where $L\, (0,0)$ (also denoted
$L_{0,0}$) is the root density at surface level on the row.  We then
define the following root distribution for a single row as
\begin{equation}
  L\, (x,z) = L_{0,0}\, e^{-a_z z} e^{-a_x |x|}
  \label{eq:Lxz}
\end{equation}
where $a_z$ and $a_x$ control the density decrease in the two
dimensions. Here, root density decrease horizontally with both depth
($z$) and horizontal distance to the row ($|x|$).

To estimate values for the parameters $a_z$, $a_x$ and $L_{0,0}$, we
assume as before that the root zone depth and mass are known, and now
additionally that the root zone width at soil surface $w_c$ is known.
We define the root zone depth $d_c$ to be the depth right below the
row ($x = 0$) where the root density is $L_m$.  As $x=0$ is the place
where \MyEq{eq:Lxz} predicts the highest density, the average root
density at $d_c$ will be less than $L_m$.  Similarly, we define the
width $w_c$ as the horizontal distance from the row where the root
density at the surface ($z = 0$) is equal to $L_m$ (Fig.~\ref{fig:row}).
\begin{equation}
  L_m = L\, (0,d_c) = L\,(w_c,0)
  \label{eq:minroot}
\end{equation}
\begin{figure}[htbp]
  \hspace{-14mm}\input{row}
  \caption{The \twod{} root zone of a single row of crops.  
    %%The x-axis
    %%denotes horizontal distance to the row, and the z-axis depth below
    %%ground level.  
    The highest root density ($L_{0,0}$) can be found
    in the row at ground level.  The root density decreases
    exponentially, with both horizontal and vertical distance.  Four
    root density isolines are shown.  The three innermost isolines
    each represents a fixed decrease in root density.  The last
    isoline represents the threshold value, $L_m$, and defines both
    the root zone depth ($d_c$) and the root zone width ($w_c$).}
  \label{fig:row}
\end{figure}

The total root length on one side of the row ($l_R$), which we assume
is known from our crop model, is the integral of the root density over
the half plane
\begin{equation}
    l_R = \int_0^{\infty} \int_0^{\infty} L\, (x,z) \: dz \, dx = \frac{L_{0,0}}{a_z a_x}
%%        = \int_0^{\infty} \int_0^{\infty} L_{0,0}\, e^{-a_z z} e^{-a_x |x|} \: dz \, dx
  \label{eq:root-integral2}
\end{equation}
Thus \MyEq{eq:Lxz} can be rewritten 
\begin{equation}
  L\, (x,z) = l_R\, a_z\, a_x \, e^{-a_z z} e^{-a_x |x|}  
  \label{eq:axz}
\end{equation}
By using \MyEq{eq:axz} in \MyEq{eq:minroot} we get
\begin{eqnarray}
  L_m &=& l_R\, a_z\, a_x \, e^{-a_z d_c} \\\label{eq:Ld0}
  L_m &=& l_R\, a_z\, a_x \, e^{-a_x w_c}\label{eq:L0r}
\end{eqnarray}
Thus $e^{-a_z d_c} = e^{-a_x w_c}$ or
\begin{equation}
  a_x = \frac{d_c}{w_c} a_z
  \label{eq:aztoax}
\end{equation}
By inserting \MyEq{eq:aztoax} in \MyEq{eq:Ld0} we get
\begin{equation}
  L_m = l_R\, a_z\, \frac{d_c}{w_c} a_z \, e^{-a_z d_c}
  \label{eq:azeq1}
\end{equation}
If we substitute 
\begin{equation}
  Q = -a_z d_c
  \label{eq:Qaz}
\end{equation}
and isolate the known values on the right side, this gives us:
\begin{equation}
  Q^2\, e^Q = L_m \frac{d_c \, w_c}{l_R}
  \label{eq:logsquare}
\end{equation}
The equation can have multiple solutions, using reasoning similar to
$W$ (\MyEq{eq:Lambert}) we find that only the solution for $Q\leq -2$
is meaningful (i.e.\@ represents a situation where the majority of the
roots are found within the root zone).  The solution can be found
using Newton's method with an initial guess of $-3$. Knowing $Q$ we
can find $a_z$~\MyEq{eq:Qaz}, $a_x$~\MyEq{eq:aztoax}, and
$L_{0,0}$~\MyEq{eq:root-integral2}.

If the rows are close, the root systems will overlap as shown on
Fig.~\ref{fig:zigzag}. In theory, all the rows on the field will
contribute some roots to the interval.  The root density in the
interval will be the sum of all the individual contributions.
\begin{figure}[htbp]
  \hspace{-6mm}\input{rootdens_L}
  \caption{The x-axis represents the distance from a specific row to
    the midpoint between it and the row to its right.  The L-axis is
    the root density for roots originating in a specific row.  The
    roots from three rows (the row itself and the rows at each side)
    is shown.}
  \label{fig:zigzag}
\end{figure}

If $D$ is the distance between rows, and we assume an infinite number
of identical rows, this can be expressed by the equation
\begin{equation}
  L^* (x,z) = \sum_{i=0}^{\infty} (L\, (x + i D,z) + L\, (D + i D - x,z))\\
  \label{eq:Lxzstar}
\end{equation}
where $L\, (x,z)$ represents the contribution from the row
itself,\linebreak{} $\sum_{i=1}^{\infty} L\, (x + i D,z)$ represents
the contributions from all the rows to the left, and
$\sum_{i=0}^{\infty} L\, (D + i D - x,z)$ represents the contributions
from all the rows to the right.  \MyEq{eq:Lxzstar} is valid for the
half row $0 \leq x \leq D/2$, but can be extended outside the interval
by assuming all rows are identical $L^* (x,z) = L^* (x+D,z)$ and
symetrical $L^* (x,z) = L^* (-x,z)$. We can rewrite \MyEq{eq:Lxzstar}
as
\begin{equation}
  \begin{array}{rl}
     & \sum_{i=0}^{\infty} (L\, (x + i D,z) + L\, (D + i D - x,z))\\
%%    =& L_{0,0}\, e^{-a_z z} 
%%       \sum_{i=0}^{\infty} (e^{-a_x (x + i D)} + e^{-a_x (D + i D - x)})\\
    =& L_{0,0}\, e^{-a_z z} (       e^{-a_x x} \sum_{i=0}^{\infty} e^{-a_x i D} 
                          + e^{-a_x (D - x)} \sum_{i=0}^{\infty} e^{-a_x i D})\\
%%    =& L_{0,0}\, e^{-a_z z} (e^{-a_x x} + e^{-a_x (D - x)})
%%       \sum_{i=0}^{\infty} e^{-a_x i D}) \\
    =& L_{0,0}\, e^{-a_z z} (e^{-a_x x} + e^{-a_x (D - x)})
       \sum_{i=0}^{\infty} ((\frac{1}{e})^{a_x D})^i\\
    =& \frac{L_{0,0}\, e^{-a_z z} (e^{-a_x x} + e^{-a_x (D - x)})}
            {1 - \frac{1}{e}^{a_x D}}\\
  \end{array}
  \label{eq:Lxzstar-solved}
\end{equation}
Two examples of the resulting root density distribution can be found
in Fig.~\ref{fig:tworows}.

\subsubsection{Mapping between the models}
\label{sec:mapping}

We can get the vertical density distribution for \MyEq{eq:Lxzstar} at
a specific depth $L^* (z)$ by finding the total root length density of
a single row \MyEq{eq:Lxz} at the specific depth, and divide with the
distance between rows.
\begin{equation}
  L^*(z) = \frac{\int_{-\infty}^{\infty} L\, (x,z)\, dx}{D}
  \label{eq:x-integrated}
\end{equation}
By setting $L^* (z) = L\, (z)$ and inserting \MyEq{eq:Lxz} and
\MyEq{eq:g+p} in \MyEq{eq:x-integrated} we get
\begin{equation}
  \begin{array}{rcl}
    L_0\, e^{-a z} &=& \frac{1}{D} \int_{-\infty}^{\infty} L_{0,0}\, e^{-a_z z} e^{-a_x |x|} dx\\
                 &=& \frac{2 L_{0,0}\, e^{-a_z z}}{D} \int_{0}^{\infty} e^{-a_x x} dx\\
%%                 &=& L_{0,0}\, e^{-a_z z} \frac{0 - 1}{-a_x}\\
                 &=& \frac{2 L_{0,0}}{D a_x} e^{-a_z z}\\
  \end{array}
  \label{eq:1d2d}
\end{equation}
hence $a = a_z$ and $L_0 = \frac{2 L_{0,0}}{a_x D}$ are the parameter
values to use in \MyEq{eq:g+p} to get the same vertical density
distribution as \MyEq{eq:Lxzstar}.

\subsubsection{Software availability}

Software to estimate root mass and root zone depth from \oned{} root
density data, and in addition root zone width from \twod{} root
density data, can be found at \url{http://www.daisy-model.org/} (look
under \texttt{Wiki}, \texttt{\textsc{gp2d}}).  Estimating of root
density distribution from root mass, root zone depth, and optionally,
root zone width, is also supported.  The program code is written in
\cplusplus{} and is covered by on open source license (\textsc{gnu
  lgpl}), and can be freely incorporated in other models.

\subsection{Data description and methodology}

Root sampling were performed at two experimental fields, the first
being operated by the Chinese Academy for Agricultural Sciences (CAAS)
and located near Zhengzhou in the Henan province, 600 km SW of
Beijing, and the second operated by the Chinese Agricultural
University and located near Beijing.  Potatoes (\textit{Solanum
  tuberosum} L. cv. 'Zheng Zha No.5') were grown at the CAAS site,
while the CAU site were growing tomatoes (\textit{Solanum
  lycopersicum} L. cv. 'Hong-fen').

\subsubsection{Treatments}

All SAFIR treatments were designed with the goal that the constituents
of the wastewater should not affect crop growth.  Hence, for crop
growth the different irrigation sources could be seen as replicates.
The second aspect was how the design of the irrigation system could
best utilize the water.  Two irrigation methods and three irrigation
strategies were considered.  The irrigation methods were subsurface
drip irrigation and furrow irrigation.  The irrigation strategies were
full irrigation (equivalent to potential evapotranspiration), deficit
irrigation (70\% of full irrigation), and partial rootzone drying
(like deficit, but application on alternating sides of the crop).

At the CAAS site, root density was successfully sampled in 2008 on one
plot for each treatment, as outlined in
Table~\ref{tab:caas2008treatments}. At CAU, root density was
successfully sampled in 2007 and 2008 at selected plots, see
Table~\ref{tab:cautreatments}.

\begin{table}[htbp]
  \caption{CAAS 2008 treatments with root density measurements.  
    All treatments used secondary treated wastewater.  Some treatments
    also used a sand filter (SF), addition of heavy metals with
    subsequent removal (HM), or ultraviolet light (UV). Plot 21 and 24 
    had no additional treatment. \citep{battilani2010decentralised}}
  \label{tab:caas2008treatments}
  \begin{tabular}{llll}\\\hline
Plot	& Source	& Method	& Strategy\\\hline
3	& SF + HM + UV	& Subsurface	& Partial rootzone drying\\
6	& SF + HM + UV	& Subsurface	& Full irrigated\\
9	& SF + HM + UV	& Subsurface	& Deficit Irrigation\\
12	& SF + HM	& Subsurface	& Deficit Irrigation\\
15	& SF + HM	& Subsurface	& Partial rootzone drying\\
18	& SF + HM	& Subsurface	& Full irrigated\\
21	& 	& Furrow	& Partial rootzone drying\\
24	& 	& Furrow	& Full irrigated
  \end{tabular}
\end{table}

\begin{table}[htbp]
  \caption{CAU treatments with root density measurements.  
    All treatments used tap water.  The strategy used was either 
    partial rootzone drying (PRD) or full irrigation (FI).}
  \label{tab:cautreatments}
  \begin{tabular}{llllll}\\\hline
\multicolumn{3}{c}{2007} & \multicolumn{3}{c}{2008}\\
Plot	& Method	& Strategy	& Plot	& Method	& Strategy\\\hline
1	& Subsurface	& PRD	& 3	& Subsurface	& PRD\\
2	& Subsurface	& PRD	& 6	& Subsurface	& FI\\
3	& Subsurface	& PRD	& 9	& Furrow	& PRD\\
4	& Subsurface	& FI	& 12	& Furrow	& FI\\
5	& Subsurface	& FI	& 15	& Subsurface	& FI\\
\end{tabular}
\end{table}

For the CAAS site, the fully irrigated treatments all had yield
between 22 and 23 Mg/ha, while the deficit and PRD treatments had
yields from 18 to 21 Mg/ha, with the exception of plot 21 where the
yield was 14 Mg/ha.  For comparison, FAOSTAT reports Chinese national
average for 2008 as 15 Mg/ha.  For the CAU site, yield were between 42
and 52 Mg/ha (Chinese average 23 Mg/ha).  There was no clear link
between yield and treatment. See~\citet{Jensen2010403} for further
discussion.

\subsubsection{Root sampling}

Both sites used ridge systems, with the crop grown on the ridges.  To
describe it, we used a coordinate system with the x-axis orthogonal to
the ridges with $x$ = 0 cm representing the top of the ridge, the
y-axis along the ridges with $y$ = 0 cm representing the plant
position, and the z-axis representing depth below the undisturbed
ground level (before ridging).  At the CAAS site, roots were sampled
in a $3\times2\times7$ grid of sample points.  The sampling along the x-axis
corresponded to the top of the ridge, the middle of the ridge wall,
and the bottom of the valley ($x$ = 0 cm, $x$ = 18.75 cm, $x$ = 37.5
cm).  The sampling along the y-axis corresponded to the plant
location, and the middle between two plants ($y$ = 0 cm, $y$ = 15 cm).
At the z-axis, there were sampling in 10 cm intervals starting just
below the undisturbed ground level ($z$ = 5 cm) at the top of the
ridge, and starting 10 and 20 cm lower for sampling at the middle and
bottom of the ridge system.  The sampling points are illustrated in
Fig.~\ref{fig:sample-caas}.  At the CAU site, the sampling was done in
a cross shape, with five sample locations at the top of the ridge ($x$
= 0 cm; and $y$ = -20 cm, $y$ = -10 cm, $y$ = 0 cm, $y$ = 10 cm, $y$ =
20 cm), and five sample locations across the ridge system ($y$ = 0 cm;
and $x$ = -30 cm, $x$ = -15 cm, $x$ = 15 cm, $x$ = 0 cm, $x$ = 30 cm).
The sampling along the top of the ridge started at $z$ = -15 cm, that
is above undisturbed ground level.  The sampling across the ridge
started at $z$ = 5 cm.  The samplings were performed in 10 cm
intervals, the lowest varied between plots and with the horizontal
distance to the plant, the lowest were at $z$ = 55 cm. See
Fig.~\ref{fig:sample-cau}.

\begin{figure}[htbp]
  \hspace{-1mm}\input{sample-caas}
  \caption{Sampling points for CAAS site.  
%%    Root density has been sampled at the center of the ridge ($x$ = 0
%%    cm), at the center of the furrow ($x$ = 37.5 cm), and halfway
%%    between the furrow and ridge ($x$ = 18.75 cm).  This sampling was
%%    done both at the plant location ($y$ = 0 cm), and between the plants
%%    in the row ($y$ = 15 cm).  The sampling depth starts at soil surface
%%    and continues in 10 cm intervals to 70 cm.  There are no samples
%%    in the top of the ridge, above the original surface level.
}
  \label{fig:sample-caas}
\end{figure}

\begin{figure}[htbp]
  \hspace{-1mm}\input{sample-cau}
  \caption{Sampling points for CAU site.  
%%    Root density has been sampled at a cross shape with center at
%%    plant location, with samples 10 and 20 cm at each side of the
%%    plant location within the ridge, and 15 and 30 cm at each side
%%    across the ridge system.  The sampling depths start 15 centimeters
%%    above the original ground level within the ridge, and continues
%%    down to 55 cm at the plant location and 10 cm away, down to 45 cm
%%    15 cm away, and down to 35 cm 20 and 30 cm away. 
    We only use the
    data from the samplings across the ridge system and below the
    original ground level, marked $+$ on the figure.}
  \label{fig:sample-cau}
\end{figure}

The CAAS root sampling was performed 2008-06-10, 16 days before
harvest.  The CAU root samplings were performed 2007-05-09 (before
flowering), 2007-07-05 (after flowering), 2008-06-15 (middle of
flowering), and 2008-07-24 (second harvest). At both sites roots were
isolated by soaking the soil samples in water in a 5 l pot and wash
gently to remove the soil particles using the flow of a
sprinkler. Several times during this process the floating roots and
organic debris were poured into a 250 $\upmu$m mesh-sizes sieve for further
cleaning and collecting the mixture of roots and organic debris. The
procedure was repeated as many times as necessary until no more roots
were left in the suspension. After removing the soil particles, the
mixture of roots and debris was transferred into a tray and roots were
separated from the organic debris using a tweezers. Root length in the
samples was determined according to \citet{tennant1975} with the
modification described by \citet{ahmadi2011} and \citet{andersen1992}
to keep the coefficient of variation around 10\% by counting at least
100 intersects for each sample

\subsubsection{Data availability}

The root density data from the two field sites is available at
\url{http://www.safir4eu.org/} (look under \texttt{Trials}. Contact
\texttt{MathiasN.Andersen@agrsci.dk} for a password).

\subsection{Fitting the model to the data}

The root density model presented in this paper assumes flat soil.  To
solve this, a virtual soil surface corresponding to the undisturbed
ground level ($z$ = 0 cm) was used in the model.  Furthermore, the
root density model is \twod{}, while the root data for both sites is
\threed{}.  The root model describes the x-axis (position between row)
and the z-axis (height above ground), but not the y-axis (position
within row).  For the CAAS dataset, we have similar datasets for two
$y$-values, and we have included both datasets, thus fitting a \twod{}
model to a \threed{} dataset.  From the point of view of the model, we
have two observations for each ($x$, $z$) pair.  For the CAU dataset,
all the observations where $y\neq$ 0 cm have $x$ = 0 cm.  Including
these would require a 3D model.  So for the CAU dataset we have only
used the measurements where $y$ = 0 cm.  Furthermore, we have ignored
the measurements with $z <$ 0 cm (above the flat soil surface of the
model), as illustrated on Fig.~\ref{fig:sample-cau}.  For each year,
we have one dataset for each plot, plus one additional dataset
consisting of all measurements from the same site, in effect a dataset
where we considered the different treatments as replicates.

We used fixed values for the minimal root density ($L_m = 0.1$
cm/cm$^3$) and specific root length ($S_r = 100$ m/g), both taken from
Daisy.  The distance between rows ($D$) were taken from the
experimental setup, 75 cm for CAAS and 80 cm for CAU.  Given these, we
could create a function that calculated the coefficient of
determination (R$^2$) for the \oned{} model based on root dry mass
($M_r$) and root zone depth ($d_c$) as described in
section~\ref{sec:model-1d}.  For the \twod{} model, we created a
similar function that in addition took the horizontal root zone width
($w_c$) as a parameter (section~\ref{sec:model-2d}).  For
each dataset and model, we found the parameter values that gave the
highest R$^2$ using the~\cite{nelder1965simplex} simplex algorithm.

The \oned{} model has two free variables ($M_r$ and $d_c$) as used
above, while the \twod{} model has three free variables ($M_r$, $d_c$,
and $w_c$).  Furthermore, as demonstrated in
section~\ref{sec:mapping}, the \oned{} model is a special case of the
\twod{} model.  This means that we can use a partial F-test
\citep{ftest} on the hypothesis that the \twod{} model provides no
significant advantage over the \oned{} model.  We used $p=0.05$ as the
test criteria.

\section{Results}

Table~\ref{tab:ftest-caas} and~\ref{tab:ftest-cau} show the best fit
root parameters for the \twod{} root density model, as well as the
coefficient of determination (R$^2$) for both the \twod{} and \oned{}
models, and the F-test value.  The horizontal root zone width is the
parameter that shows the largest variation between treatments, and
root zone depth is the parameter that shows least variation.  With the
exception of the early 2008 sampling of CAU plot 12, the \twod{} R$^2$
is always above 0.55, with a median of 0.86.  The potato (CAAS) tend
to have a wider root zone and more root mass than the tomato (CAU),
but no clear difference in root zone depth.  The \oned{} R$^2$ were
much lower (less than 0.20) for the plots where the estimated \twod{}
root zone width was less than half the distance between rows.  The
F-test showed that the \twod{} model gave a significant better fit
than the \oned{} model for all the aggregate datasets, half the CAAS
datasets, and all but three of the CAU datasets.

\begin{table}[htbp]
  \caption{The best fit for root zone depth ($d_c$),
    root zone diameter ($2 w_c$), and total root dry matter ($M_r$) is show
    for each plot at CAAS.  The `All' plot indicate a dataset containing all
    the individual plots.  Furthermore, the R$^2$ for both the \oned{}
    and \twod{} fits are listed, as well as an F test indicating whether
    the \twod{} model provides significantly better fit (\textbf{bold}).} 
  \label{tab:ftest-caas}
  \begin{tabular}{rrrrrrrrr}\\\hline
Plot	& $d_c$	& $2\;w_c$ & $M_r$ & \twod{} & \oned{} & F	& F \\
	& cm	& cm	& Mg/ha	& R$^2$	& R$^2$	& Obs	& 0.05 \\\hline
\multicolumn{8}{c}{2008-06-10 CAAS} \\
All	& 93	& 216	& 0.65	& 0.61	& 0.60	& \textbf{5.42}	& 3.88\\
3	& 91	& 265	& 1.22	& 0.93	& 0.92	& 2.80	& 4.14\\
6	& 81	& 743	& 1.11	& 0.88	& 0.88	& 0.03	& 4.14\\
9	& 106	& 151	& 0.52	& 0.94	& 0.89	& \textbf{25.23}	& 4.14\\
12	& 88	& 151	& 0.39	& 0.94	& 0.91	& \textbf{17.75}	& 4.14\\
15	& 87	& $\infty$	& 0.91	& 0.75	& 0.75	& 0.00	& 4.14\\
18	& 87	& 82	& 0.22	& 0.88	& 0.79	& \textbf{23.80}	& 4.14\\
21	& 88	& 99	& 0.28	& 0.86	& 0.79	& \textbf{16.53}	& 4.14\\
24	& 132	& 164	& 0.74	& 0.73	& 0.69	& 4.10	& 4.14\\
  \end{tabular}
\end{table}

\begin{table}[htbp]
  \caption{The best fit for root zone depth ($d_c$),
    root zone diameter ($2 w_c$), and total root dry matter ($M_r$) is show
    for each plot at CAU.  The `All' plot indicate a dataset containing all
    the individual plots.  Furthermore, the R$^2$ for both the \oned{}
    and \twod{} fits are listed, as well as an F test indicating whether
    the \twod{} model provides significantly better fit (\textbf{bold}).} 
  \label{tab:ftest-cau}
  \begin{tabular}{rrrrrrrrr}\\\hline
Plot	& $d_c$	& $2\;w_c$ & $M_r$ & \twod{} & \oned{} & F	& F \\
	& cm	& cm	& Mg/ha	& R$^2$	& R$^2$	& Obs	& 0.05 \\\hline
\multicolumn{8}{c}{2007-05-30 CAU} \\
All	& 78	& 96	& 0.18	& 0.63	& 0.49	& \textbf{43.69}	& 3.93\\
1	& 56	& 48	& 0.11	& 0.84	& 0.40	& \textbf{53.28}	& 4.38\\
2	& 83	& 75	& 0.17	& 0.76	& 0.46	& \textbf{23.72}	& 4.38\\
3	& 83	& 271	& 0.23	& 0.84	& 0.83	& 1.00	& 4.32\\
4	& 90	& 259	& 0.24	& 0.83	& 0.82	& 1.09	& 4.32\\
5	& 72	& 47	& 0.15	& 0.96	& 0.31	& \textbf{309.76}	& 4.38\\\hline
\multicolumn{8}{c}{2007-07-05 CAU}\\
All	& 112	& 64	& 0.33	& 0.67	& 0.19	& \textbf{172.94}	& 3.92\\
1	& 85	& 39	& 0.20	& 0.88	& 0.16	& \textbf{121.21}	& 4.32\\
2	& 94	& 40	& 0.22	& 0.94	& 0.15	& \textbf{276.04}	& 4.32\\
3	& 137	& 97	& 0.51	& 0.76	& 0.34	& \textbf{36.52}	& 4.32\\
4	& 126	& 89	& 0.50	& 0.84	& 0.32	& \textbf{67.25}	& 4.32\\
5	& 93	& 39	& 0.21	& 0.94	& 0.15	& \textbf{265.82}	& 4.35\\\hline
\multicolumn{8}{c}{2008-06-15 CAU}\\
All	& 124	& 63	& 0.27	& 0.56	& 0.14	& \textbf{85.36}	& 3.95\\
3	& 72	& 31	& 0.14	& 0.89	& 0.14	& \textbf{96.81}	& 4.54\\
6	& 99	& 69	& 0.23	& 0.86	& 0.37	& \textbf{54.02}	& 4.54\\
9	& 144	& 63	& 0.30	& 0.83	& 0.15	& \textbf{58.11}	& 4.54\\
12	& 245	& 122	& 0.63	& 0.30	& 0.11	& 4.09	& 4.54\\
15	& 125	& 44	& 0.24	& 0.85	& 0.12	& \textbf{70.26}	& 4.54\\\hline
\multicolumn{8}{c}{2008-07-24 CAU}\\
All	& 108	& 45	& 0.25	& 0.92	& 0.13	& \textbf{894.32}	& 3.95\\
3	& 120	& 42	& 0.28	& 0.95	& 0.06	& \textbf{260.20}	& 4.60\\
6	& 95	& 44	& 0.22	& 0.96	& 0.17	& \textbf{281.54}	& 4.54\\
9	& 115	& 53	& 0.28	& 0.94	& 0.15	& \textbf{184.84}	& 4.54\\
12	& 101	& 40	& 0.23	& 0.96	& 0.17	& \textbf{282.93}	& 4.60\\
15	& 113	& 47	& 0.26	& 0.89	& 0.11	& \textbf{101.73}	& 4.60\\
  \end{tabular}
\end{table}

The aggregate datasets have an R$^2$ for the \twod{} model that is
lower than for the individual datasets, except for the two 2008 CAU
samplings.  We see on Fig.~\ref{fig:caas2008}, \ref{fig:cau2007},
and~\ref{fig:cau2008} that the CAU 2008 aggregate datasets also have
the lowest standard deviation between plots.  The CAAS 2008 aggregate
dataset has the highest standard deviation, reflecting the fact that
half the model parameter fits show a wide root zone, and the other
half a narrow root zone.  Both plots using subsoil drip irrigation and
partial rootzone drying for the CAAS site show a wide root zone, and
both plots with subsoil drip and ``normal'' deficit irrigation show a
relatively narrow root zone.  However, one of the fully irrigated
subsoil plots has a wide root zone, and the other a narrow root zone.
The three samplings at the CAU site that shows a wide root zone
represent both irrigation methods and both irrigation strategies.  The
two 2008 sampling at CAU shows a tendency of the roots to concentrate
near the center of the row, as illustrated on Fig.~\ref{fig:tworows}.
The tendency is partly supported by the 2007 data, where the root zone
sampling was performed earlier.

\begin{figure}[htbp]
  \input{CAAS-2008-0}\\
  \input{CAAS-2008-18}\\
  \input{CAAS-2008-37}
  \caption{Estimated and observed root density for the 2008 CAAS root
    density sampling at three different distances from the row.  
    %% The top graph show data from the plant row, the
    %% center graph 18.75 cm from the row, and the bottom graph 37.50 cm
    %% from the plant rows.  The z-axes represent depth below original
    %% ground level, the x-axes represent root density.  The curvy lines
    %% are the modelled root density.  
    The error bars represent mean and standard deviation for all treatments}
  \label{fig:caas2008}
\end{figure}

\begin{figure*}[htbp]
  \input{CAU-2007a-0}\input{CAU-2007b-0}\\
  \input{CAU-2007a-15}\input{CAU-2007b-15}\\
  \input{CAU-2007a-30}\input{CAU-2007b-30}
  \caption{Estimated and observed root density for the early and late
    2007 CAU root density samplings at three different distances from the row.  
    %% The left side is the May 30
    %% sampling, the right side the July 5 sampling.  The top graphs show
    %% data from the plant row, the center graphs 15 cm from the row, and
    %% the bottom graphs 30 cm from the plant rows.  The z-axes represent
    %% depth below the original ground level, the x-axes represent root
    %% density.  The curvy lines are the modelled root density.  
    The error bars represent mean and standard deviation for all
    treatments, including measurements at both sides of the row.}
  \label{fig:cau2007}
\end{figure*}

\begin{figure*}[htbp]
  \input{CAU-2008a-0}\input{CAU-2008b-0}\\
  \input{CAU-2008a-15}\input{CAU-2008b-15}\\
  \input{CAU-2008a-30}\input{CAU-2008b-30}
  \caption{Estimated and observed root density for the early and late
    2008 CAU root density samplings at three different distances from the row.
    %% The left side is the June 15
    %% sampling, the right side the July 24 sampling.  The top graphs
    %% show data from the plant row, the center graphs 15 cm from the
    %% row, and the bottom graphs 30 cm from the plant rows.  The z-axes
    %% represent depth below the original ground level, the x-axes
    %% represent root density.  The curvy lines are the modelled root
    %% density.  
    The error bars represent mean and standard deviation for
    all treatments, including measurements at both sides of the row.}
  \label{fig:cau2008}
\end{figure*}

\begin{figure}[htbp]
  \hspace{-4mm}\vbox{\input{CAU2008a}\\
    \input{CAU2008b}}
  \caption{Estimated root zone for the CAU site for the two 2008 root
    samplings.
    %%The top graph shows the 15-06-2008 root density
    %%distribution and the bottom graph the 24-07-2008 root density
    %% distribution. 
    Two interacting rows at $x$ = 0 and 80 cm are shown.  The lines
    represent root density ($L$) isolines.}
  \label{fig:tworows}
\end{figure}

\section{Discussion and concluding remarks}

\subsection{New}
We have presented a simple and robust \twod{} root density model for
row crops, which supports overlapping root zones from multiple rows,
and have shown how the model can be parametrized based on physically
meaningful quantities. We have also shown that the \twod{} model is a
generalized version of a well know and tested \oned{} model.

To test the new model, we have fitted both the original \oned{} model
and the generalized \twod{} model to datasets spanning two years data
from two sites (CAAS and CAU) with different crops, and performed a
statistical test to determine if the \twod{} model can provide a
significantly better fit than the \oned{} model.

Both sites were using ridge systems and had \threed{} datasets. Our
\twod{} model, which assumes flat soil, were thus not an obvious
match. However, since the root density model was intended to be used
in conjunction with Daisy, and the soil module of Daisy is \twod{} and
assumes a flat soil surface, the root density model had to have the
same properties. However the geometry mapping represents an additional
uncertainty, especially in conjunction with the estimated physical
quantities, $d_c$, $w_c$, and $M_r$.

For the CAAS datasets, the \twod{} model fit in general estimates a
wide root zone, which we believe explains why the \oned{} model can
provide a good match (R$^2 \geq 0.69$ for the individual plots). The
R$^2$ values for the \twod{} are always equal or higher than for the
\oned{} model, as must be the case one model is a generalization of
another. For half the individual plots the improvement is
statistically significant. When looking at dataset containing all
plots, the improvement is small (0.61 vs 0.60), but significant due to
the number of data points. For the CAU datasets, a much comparable
narrow root zone is estimated by the \twod{} model fit, and as a
result, we generally get a worse fit for the \oned{} model.

The \twod{} model performs significantly better for most individual
plots, and for all combined datasets. Since the combined datasets
contain observations from multiple plots, they are less susceptible to
random variations in local soil conditions, and therefore more
trustworthy. Furthermore, visual inspection of observed data vs.\@
model fit (Fig.~\ref{fig:caas2008},~\ref{fig:cau2007},
and~\ref{fig:cau2008}) shows that the model provides a good
description of both the horizontal and vertical variation in root
density.

The root distribution is changing over the lifetime of the crop. We
expert the \twod{} model in general to perform better in the early
vegetative phase where the root system is not yet fully developed, and
for some crops based on the CAU results also in the late reproductive
phase. However, for a fully developed root zone the \oned{} model may
be adequate to describe the root density distribution. More temporal
data points would be needed to confirm this.

We see the primary application of the \twod{} model to be when
modelling water or solute uptake from row crops. The use of the
\twod{} model may only be relevant for the early and perhaps late
phase of the crop development. Even though the improvement provided by
the \twod{} model for the CAAS data was statistically significant, it
is plausible that the precision of the \oned{} model (R$^2$ = 0.60)
would be adequate at this stage of the crop development.  

The secondary application we see is inverse modelling, where you go
from root density sampling, to the model parameters, as we have done
in this paper. This approach provides a large data reduction, which
can be useful for getting an overview of the results.

Compared to more advanced models, the main advantage of the simple
\twod{} model described in this paper is that it can be parametrised
based on very few data, namely estimates of root mass, and root zone
depth and width, so it is useful in the common case where limited root
data is available.  For a project where the focus is specifically on
the root development rather than above ground production or water and
nitrogen balances, the model is inadequate.  In particular, a dynamic
root density model would be needed to explore how the development is
affected by the local soil conditions, such as mechanical resistance,
or availability of water and nutrients.

\subsection{Old}


A major source of uncertainty is the remapping of the 3D ridge
geometry to a 2D flat surface.  For both sites, we ignore the root
mass that is placed in the top of the ridge.  We also ignore that the
ridge valley does not contain any soil.  For the CAU site we
furthermore ignore the third dimension, using only measurements at y =
0 cm where the root density is highest.  These systematic errors
affect the estimate of all three parameters, but especially the root
dry mass.  However, a comparison between plots at the same site should
still be valid.

The statistical analysis (Table~\ref{tab:ftest-caas}
and~\ref{tab:ftest-caas}) shows us that the presented \twod{} model
constitutes a significant improvement over the \oned{} model,
especially in the cases where the root zone from the individual row is
narrow.  Furthermore, visual inspection of observed data vs.\@
predicted data (Fig.~\ref{fig:caas2008},~\ref{fig:cau2007},
and~\ref{fig:cau2008}) shows that the model provides a good match for
both the horizontal and vertical variation in root density.

Additionally, we believe that conversion of root density measurements
to general crop parameters (root dry mass, root zone depth, and root
zone width) will be useful when analysing the effect of treatments,
climate, or soil type on root development.  Such an analysis is
outside the scope of this paper, but the numbers presented on
Table~\ref{tab:ftest-cau} and~\ref{tab:ftest-caas} do not give a clear
indication of the difference in treatments being the main factor in
the difference between plots for the present experiment.

In conclusion, we managed to introduce a simple \twod{} extension to
the \oned{} descriptive root density model of \citet{gp74},
reformulate both the original and the extended models for meaningful
physical quantities suitable for use with mechanistic crop growth
models, and show that the extended model gives a significantly better
match for row crops than the original for two independent datasets.

\section*{Acknowledgements}

The research was partly funded by the EU-DG XII (FP6 Contract No. CT
Food-2005-023168 SAFIR).

\section*{References}

\bibliographystyle{elsart-harv}
\bibliography{../daisy}

\vfill{}\section*{List of symbols}

\begin{tabular}{lll}
Symbol  & Unit    & Description\\\hline
$a$     & L$^{-1}$ & Root density distribution parameter\\
$a_z$   & L$^{-1}$ & Vertical root density distribution\\
$a_x$   & L$^{-1}$ & Horizontal root density distribution\\
$d_c$   & L       & Crop potential root zone depth\\
$D$     & L       & Distance between rows\\
$l_r$   & L/L$^2$ & Total root length per area\\ 
$l_R$   & L/L     & Total root length per length of row\\ 
$L_0$   & L/L$^3$ & Average root density at soil surface\\
$L_{0,0}$& L/L$^3$ & Root density in row at soil surface\\
$L_m$   & L/L$^3$ & Minimal root density\\ 
$L\, (z)$   & L/L$^3$ & Root density at soil depth $z$\\
$L^\prime (z) $ & L/L$^3$ & Soil limited root density\\ 
$L\, (x,z)$& L/L$^3$ & Root density at position $(x,z)$\\
$L^* (x,z)$& L/L$^3$ & Root density from multiple rows\\
$L^* (z)$& L/L$^3$ & Average vertical root density for row\\
$M_r$   & M/L$^2$& Total root dry matter\\ 
$Q$     &         & Substitution variable\\
$S_r$   & L/M    & Specific root length\\ 
$w_c$     & L     & Horizontal root zone width\\
$W$     &         & Substitution variable\\ 
$x$     & L       & Horizontal position across row \\
$y$     & L       & Horizontal position along row \\
$z$     & L       & Soil depth \\
\end{tabular}
\vfill{}

\end{document}
