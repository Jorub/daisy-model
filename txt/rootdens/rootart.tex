\documentclass[a4paper,11pt,twoside]{article}
\usepackage{a4}
\usepackage[T1]{fontenc}
\usepackage[latin1]{inputenc}
\usepackage{amsmath}
\usepackage{hyperref}
\usepackage{natbib}
\usepackage{graphicx}
\newcommand{\daisy}{{\sc daisy}}
\newcommand{\Daisy}{{\sc Daisy}}
\newcommand{\cplusplus}%
{{\leavevmode{\rm{\hbox{C\hskip -0.1ex\raise 0.5ex\hbox{\tiny ++}}}}}}
\newcommand{\Cplusplus}{\cplusplus}
\newcommand{\mshe}{Mike/\textsc{she}}
\newcommand{\wintel}{\texttt{win32}}
\newcommand{\dll}{\textsc{dll}}
\newcommand{\Dll}{\textsc{Dll}}
\newcommand{\gui}{\textsc{gui}}
\newcommand{\Gui}{\textsc{Gui}}
\newcommand{\unix}{Unix}
\newcommand{\dhi}{\textsc{dhi}}
\newcommand{\Dhi}{\textsc{Dhi}}
\newcommand{\api}{\textsc{api}}
\newcommand{\Api}{\textsc{Api}}
%\newcommand{\url}[1]{\linebreak[4]\texttt{<URL:#1>}}

%%% Local Variables: 
%%% mode: latex
%%% TeX-master: t
%%% End: 


% MMO
% L_z -> L(z)
% L_{z,x} -> L(x,z)

\newcommand{\attmark}[1]{\mbox{$^{\mbox{\footnotesize #1}}$}}
\newcommand{\twod}{\textsc{2d}}
\newcommand{\oned}{\textsc{1d}}
\newcommand{\threed}{\textsc{3d}}
\begin{document}

\title{A simple, parametric \twod{} root density distribution model
  for row crops.}  

\author{Per Abrahamsen\attmark{a,}\footnote{Corresponding author.
    Present address: IGM LIFE KU, Thorvaldsensvej 40, DK-1871
    Frederiksberg.  Email: pa@life.ku.dk}, 
  Mathias Neumann Andersen\attmark{b}, Xuebin Qi\attmark{c},\\
  Guitong Li\attmark{d}, Mikkel Mollerup\attmark{a},
  and S�ren Hansen\attmark{a}\\
  \\
  \small\em\attmark{a}University of Copenhagen, Denmark\\
  \small\em\attmark{b}Aarhus University,
  Denmark\\
  \small\em\attmark{c}Chinese Academy of Agricultural Sciences, China\\
  \small\em\attmark{d}China Agricultural University, China}

\maketitle

\begin{abstract}
  TODO: Intro (From PRD to 2D root modelling)\\
  In this paper we extend an emperical root density distribution based
  on densely populated homogenious fields to row crops.  The row crops
  are modeled as having a uniform density in the direction parallel to
  the rows, but variable in the direction perpindicular to the row.
  In each case we demonstrate how to find the distribution parameters
  from the root dry matter and the size of the root zone.\\
  TODO: Data, results, and conclusion.
\end{abstract}

\section{Introduction}

\begin{itemize}
\item TODO: PRD er godt
\item TODO: Modellering af PRD kr�ver jordfysisk beskrivelse af vand
  bev�gelse i 2 dimensioner, samt en 2D model for rodt�thed (fokus for
  denne artikel).
\item TODO: Om SAFIR \cite{safir2010}.
\item TODO: Om Daisy. Udvikling af 2D: \cite{1dvs2d}.  Brug i PRD: \cite{ssoc}
\item TODO: State-of-the-art rodmodellering.  Motivation for GP2D.
\end{itemize}

\section{Model theory}

Our goal is to describe the two dimensional root density distribution
for row crops in terms of root dry mass, root zone depth, and root
zone width.  These values can be provided by a general crop growth
model (like the one found in Daisy), and furthermore have the
advantage that they are easily understood concepts.  However, first we
will describe for a traditional one dimensional root density
distribution in terms of root dry mass and root zone width.  This will
provide a reference to compare with later.

\subsection{Densily populated fields}

In accordance with \cite{gp74}, the root density distribution $L_z$
for a crop can be described by
\begin{equation}
  L_z = L_0\, e^{-a z}
  \label{eq:g+p}
\end{equation}
where $L_0$ is the root density at the soil surface, $a$ is a
distribution parameter, and $z$ is the depth below soil surface.

We here assume that the density is uniformly distributed on the
horizontal plane, an assumption that fails with e.g.\ row crops.

The parameters $a$ and $L_0$ will both vary with time.  For a
production oriented simulation model like
Daisy~\citep{daisy-fertilizer,daisy-ems}, it can be more convenient to
specify the density in terms of accumulated root dry matter $M_r$ and
total root depth $d_c$, as described in \cite{daisy-def} or the
following.

We define the root depth at the lowest depth where the root density is
at above specified threshold $L_m$.  By inserting this in
\eqref{eq:g+p}, we get
\begin{equation}
  L_m = L_{d_c} = L_0\, e^{-a d_c}
  \label{eq:root-depth}
\end{equation}

We convert the root mass to root length $l_r$ by assuming the specific
root length $S_r$ is a known constant (rather than varying with depth)
\begin{equation}
  l_r = S_r \, M_r
  \label{eq:root-length}
\end{equation}

The total root length is also the integral of the root density over
the profile
\begin{equation}
  l_r = \int_0^{\infty} L_z \: dz 
      = \int_0^{\infty} L_0\, e^{-a z} \, dz 
      = \frac{L_0}{a}
  \label{eq:root-integral}
\end{equation}

By inserting the expression we get for $L_0$ from
\eqref{eq:root-integral} in \eqref{eq:root-depth} we get
\begin{equation}
  L_m = l_r \, a \, e^{-a d_c}
  \label{eq:a-only}
\end{equation}

If we substitute $W = -a d_c$ and isolate the known values on the right
side this gives us
\begin{equation}
   W e^W = - L_m \frac{d_c}{l_r}
  \label{eq:Lambert}
\end{equation}
The solution to this equation with regard to $W$ happens to be the
definition of the Lambert-W function \citep{euler83,lambert58}.  The
function on the left hand side of the equation is depicted on
figure~\ref{fig:W}. 
\begin{figure}[htbp]
  \input{rootdens_W}
  \caption{$W e^W$}
  \label{fig:W}
\end{figure}

Since we now know the value for $W$m we can find the desired density
parameters $L_0$ and $a$ by substituting back
\begin{eqnarray}
   a   &=& -W / d_c\\\label{eq:a-solved}
   L_0 &=& \frac{L_m}{e^{-a d_c}} = L_m e^{a d_c}\label{eq:L0-found}
\end{eqnarray}

\subsubsection{Numeric solution to $W$}

We start by dividing the funtions into monotonic intervals by finding
the derivative
\begin{equation}
  \frac{d W\, e^W}{dW} = e^W + W e^W
  \label{eq:derived-W}
\end{equation}
The equation
\begin{equation}
  e^W + W e^W = 0
  \label{eq:derived-W-solutions}
\end{equation}
has one solution, $W=-1$. The expression $W\, e^W$ is decreasing below
$-1$ and increasing above $-1$.  Thus, $W=0$ is a global minimum.

Since $\lim_{Q\to-\infty} W\,e^W=0$ we get a single solution when
$-L_m \frac{d_c}{l_r}$ is exactly at the bottom point ($-1 e^{-1}$), two
when it is above (it is never positive), and none when it is below.
The later situation corresponds to the case where there are
insufficent root $l_r$ to satisfy the minimal root density $L_m$
within the given root zone $d_c$.

Both solutions are valid, but represent different distributions.
\begin{itemize} 
\item The solution for $W < -1$ represents a large $a$ parameter. From
  \eqref{eq:L0-found} we see this also means $L_0$ is large.  Thus,
  the solution corresponds to a root zone with a high density near the
  top that decreases rapidly to $L_m$ at the bottom of the root zone,
  and continues to decrease so only a small contribution to the total
  root length from below he root zone.
\item The solution for $W > -1$ (and thus small values of $a$ and
  $L_0$) corresponds to a low root density near the top that
  decreases slowly, and thus gives a larger contribution to the total
  root length from below the root zone.
\end{itemize}
As the total root length increases, pressing $W$ towards $0$ or
$-\infty$, the difference between the solutions grow.  When there is
just enough roots to satisfy the contraints at $W = -1$, the two
solutions converges to one.  As we like our roots to stay mostly
within the root zone, we choose the solution for $W < -1$.  We can
thus find $W$ numerically using Newton's method and an initial guess
of $-2$.

\subsubsection{Limited growth}

The distribution in \eqref{eq:g+p} implies a gradual decrease of roots
going towards, but never reaching zero.  There are two problems with
this.  The first one is emprical, for some soils it doesn't match what
we observe, rather than a gradual decrease, there is sharp decrease at
a specific depth, as the roots are unable to penetrate further
down. The other one is practical, too large a root zone makes
computation impractical.

The way we model the first issue is to divide the root depth into a
crop specific and soil independent potential root depth $d_c$, and soil
specific and crop independent maximum root depth $d_s$.  The actual
root depth $d_a$ is then the shallowest of these two.
\begin{equation}
  d_a = \min (d_c, d_s)
  \label{eq:actual-depth}
\end{equation}

We now create a modified root density function $L_z^*$ by defining it to
zero below $d_a$, and a $L_z$ scaled to preserve mass balance above.
\begin{equation}
  L_z^* =
  \begin{cases}
    k^* L_z & \text{if $z \leq d_a$}\\
    0 & \text{if $z > d_a$}
  \end{cases}
  \label{eq:limited-depth}
\end{equation}
where
\begin{equation}
  k^* = \frac{l_r}{\int_0^{d_a} L_z \, dz}
  \label{eq:scale-factor}
\end{equation}
thus solving both problems.

\subsection{Row crops}

We can describe a row crop with a two dimensional model by assuming
that the plants are densily packed in the row.  Our second dimension
$x$ is horizontal, ortogonal to the row.  The root density at a
specific point can be denoted $L_{z,x}$, and we chooce origo so
$L_{0,0}$ is the the root density in the top of the row.  See
figure~\ref{fig:row-crop}.
\begin{figure}[htbp]
  TODO: Insert figure here.
  \caption{Crop row.}
  \label{fig:row-crop}
\end{figure}

We then define the following root dstribution
\begin{equation}
  L_{z,x} = L_{0,0}\, e^{-a_z z} e^{-a_x x}
  \label{eq:Lzx}
\end{equation}
where $a_z$ and $a_z$ control the density decrease in the two
dimensions.

\subsubsection{Finding the parameters}

To find the parameters $a_z$, $a_x$ and $L_{0,0}$, we assume as before
that the root depth and root mass is known, and now additionally that
the root radius $w_c$ is known.  We define the root zone depth $d_c$
to be the depth right below the row ($x = 0$) where the root density
is $L_m$.  As $x=0$ is the place where \eqref{eq:Lzx} predicts the
highest density, the average root density at that depth will be well
below $L_m$.  Similarily, we define the radius $w_c$ as the horizontal
distance from the row where the root density at the surface ($z = 0$)
\begin{equation}
  L_m = L_{d,0} = L_{0,r}
  \label{eq:minroot}
\end{equation}

The total root length on one side of the row ($l_R$), which we assume
is known from our crop model, is the integral of the root density over
the half plane
\begin{equation}
  \begin{array}{rcl}
    l_R &=& \int_0^{\infty} \int_0^{\infty} L_{z,x} \: dz \, dx \\
        &=& \int_0^{\infty} \int_0^{\infty} L_{0,0}\, e^{-a_z z} e^{-a_x x} \: dz \, dx\\
        &=& \frac{L_{0,0}}{a_z a_x}\\ 
  \end{array}
  \label{eq:root-integral2}
\end{equation}
Thus \eqref{eq:Lzx} can be rewritten 
\begin{equation}
  L_{z,x} = l_R\, a_z\, a_x \, e^{-a_z z} e^{-a_x x}  
  \label{eq:azx}
\end{equation}

By using \eqref{eq:azx} in \eqref{eq:minroot} we get
\begin{eqnarray}
  L_m &=& l_R\, a_z\, a_x \, e^{-a_z d} \\\label{eq:Ld0}
  L_m &=& l_R\, a_z\, a_x \, e^{-a_x r}\label{eq:L0r}
\end{eqnarray}
Thus $e^{-a_z d} = e^{-a_x r}$ or
\begin{equation}
  a_x = \frac{d_c}{w_c} a_z
  \label{eq:aztoax}
\end{equation}

By inserting \eqref{eq:aztoax} in \eqref{eq:Ld0} we get
\begin{equation}
  L_m = l_R\, a_z\, \frac{d}{r} a_z \, e^{-a_z d}
  \label{eq:azeq1}
\end{equation}

If we substitute 
\begin{equation}
  Q = -a_z d
  \label{eq:Qaz}
\end{equation}
and isolate the known values on the right side, this gives us:
\begin{equation}
  Q^2\, e^Q = L_m \frac{d \, r}{l_R}
  \label{eq:logsquare}
\end{equation}
The left hand side expression is illustrated in figure~\ref{fig:Q}.
Unlike \eqref{eq:Lambert}, nobody bothered to give the solution to
\eqref{eq:logsquare} a name.
\begin{figure}[htbp]
  \input{rootdens_Q}
  \caption{$Q^2 e^Q$}
  \label{fig:Q}
\end{figure}

\subsubsection{Numeric solution to $Q$}

We start by dividing the funtions into monotonic intervals by finding
the derivative
\begin{equation}
  \frac{d (Q^2\, e^Q)}{dQ} = 2 Q e^Q + Q^2 e^Q
  \label{eq:derived}
\end{equation}
The equation
\begin{equation}
  2 Q e^Q + Q^2 e^Q = 0
  \label{eq:derived-solutions}
\end{equation}
has two solutions, $Q=0$ and $Q=-2$, and the expression $Q^2\, e^Q$ is
increasing below $-2$, decreasing between $-2$ and $0$, and increasing
above $0$. Thus, $Q=0$ is a local (and in this case also global)
minimum, and $Q=-2$ is a local maximum.

We are not interested in positive values for $Q$, they correspond to
negative values for $a_z$, and the simplification in
\eqref{eq:root-integral2} are only valid if $a_z > 0$.  

Since $\lim_{Q\to-\infty} Q^2\,e^Q=0$ we get a single negative
solution when $L_m \frac{d \, r}{l_R}$ is exactly at the top point
($2^2 e^{-2}$), two when it is smaller (it is never negative), and
none when it is larger.  The later situation corresponds to the case
where there are insufficent root $l_R$ to satisfy the minimal root
density $L_m$ within the given root zone $d\, r$. 

Both negative solutions are valid, but represent different
distributions.
\begin{itemize} 
\item The solution for $Q < -2$ represents a large $a_z$ (and thus
  also $a_x$) parameter. From \eqref{eq:root-integral2} we see this
  also means $L_{0,0}$ is large.  Thus, the solution corresponds to a
  root zone with a high density near the center that decreases rapidly
  to $L_m$ near the edge of the root zone, and continues to decrease
  so only a small contribution to the total root length fro outside
  the root zone.
\item The solution for $Q > -2$ (and thus small values of $a_x$, $a_x$
  and $L_{0,0}$) corresponds to a low root density near the center
  that decreases slowly, and thus gives a larger contribution to the
  total root length from outside the root zone.  
\end{itemize}
As the total root length increases, pressing $Q$ towards $0$ or
$-\infty$, the difference between the solutions grow.  When there is just
enough roots to satisfy the contraints at $Q = -2$, the two solutions
converges to one.  As we like our roots to stay mostly within the root
zone, we choose the solution for $Q < -2$.

We can find $Q$ numerically using Newton's method and an initial guess
of $-3$.  From that we can find $a_z$ from \eqref{eq:Qaz}, $a_x$ from
\eqref{eq:aztoax}, and $L_{0,0}$ from \eqref{eq:root-integral2}.

\subsubsection{Multiple rows}

If the rows are close enough, the root systems will overlap as shown
on figure~\ref{fig:zigzag}.
\begin{figure}[htbp]
  \input{rootdens_L}
  \caption{The x-axis represents the distance from a row to the
    midpoint between it and the row to its right.  The y-axis is the
    root density for roots originating in a specific row.  The top
    line represents the roots from the row itself.  The next line the
    roots from the row to the right.  And the last line the roots from
    the row to the left.  In theory, all the rows on the field will
    contibute some roots to the interval.  The root density in the
    interval will be the sum of all the individual contributions.}
  \label{fig:zigzag}
\end{figure}

If $R$ is the distance between rows, and we assume an infinite number
of identical rows, this can be expressed by the equation
\begin{equation}
  L^*_{z,x} =
    \begin{cases}
       \sum_{i=0}^{\infty} (L_{z,x + i R} + L_{z,R + i R - x}) & \text{if $x < R/2$}\\
                                                    0  & \text{if $x \geq R/2$}
    \end{cases}
  \label{eq:Lzxstar}
\end{equation}

Using \eqref{eq:Lzx} and the rules for geometric serieses we can rewrite
the first case to get rid of the sum
\begin{equation}
  \begin{array}{rl}
     & \sum_{i=0}^{\infty} (L_{z,x + i R} + L_{z,R + i R - x})\\
%%    =& L_{0,0}\, e^{-a_z z} 
%%       \sum_{i=0}^{\infty} (e^{-a_x (x + i R)} + e^{-a_x (R + i R - x)})\\
    =& L_{0,0}\, e^{-a_z z} (       e^{-a_x x} \sum_{i=0}^{\infty} e^{-a_x i R} 
                          + e^{-a_x (R - x)} \sum_{i=0}^{\infty} e^{-a_x i R})\\
%%    =& L_{0,0}\, e^{-a_z z} (e^{-a_x x} + e^{-a_x (R - x)})
%%       \sum_{i=0}^{\infty} e^{-a_x i R}) \\
    =& L_{0,0}\, e^{-a_z z} (e^{-a_x x} + e^{-a_x (R - x)})
       \sum_{i=0}^{\infty} ((\frac{1}{e})^{a_x R})^i\\
    =& \frac{L_{0,0}\, e^{-a_z z} (e^{-a_x x} + e^{-a_x (R - x)})}
            {1 - \frac{1}{e}^{a_x R}}\\
  \end{array}
  \label{eq:Lzxstar-solved}
\end{equation}

\subsubsection{Mapping between the models}
\label{sec:mapping}

We would like to retain our original distribution when ignoring the x
dimension.  We couldn't do that when looking only at the root system
for a single row, as it is infinitely wide and thus has an average
density of zero.  However, if we look at the roots of single row, we
get
\begin{equation}
  L_z = \frac{2 \int_0^{\infty} L_{z,x}\, dx}{R}
  \label{eq:x-integrated}
\end{equation}
We multiply by two as we assume the two sides of the rows are
identical.  By integrating to $\infty$ rather than just $R/2$ we do
include roots from outside the row.  However, because the system has
an infinite number of identical rows, the amount of roots from the
crop outside its own row is exactly the same as the amount of roots
from other rows inside the row we are examining.

Inserting \eqref{eq:Lzx} and \eqref{eq:g+p} in \eqref{eq:x-integrated} we get
\begin{equation}
  \begin{array}{rcl}
    L_0\, e^{-a z} &=& \frac{2}{R} \int_{0}^{\infty} L_{0,0}\, e^{-a_z z} e^{-a_x x} dx\\
                 &=& \frac{2 L_{0,0}\, e^{-a_z z}}{R} \int_{0}^{\infty} e^{-a_x x} dx\\
                 &=& L_{0,0}\, e^{-a_z z} \frac{0 - 1}{-a_x}\\
                 &=& \frac{2 L_{0,0}}{R a_x} e^{-a_z z}\\
  \end{array}
  \label{eq:1d2d}
\end{equation}
So we get
\begin{eqnarray}
  a_z &=& a\\\label{eq:azisa}
  L_{0,0} &=& � a_x R L_0\\\label{eq:L00L0}
  L_0 &=& \frac{2 L_{0,0}}{a_x R}\label{eq:L0L00}
\end{eqnarray}
as the equation to use when switching between the one and two
dimensional descriptions.

\section{Data description and methodology}

\begin{itemize}
\item TODO: Two sites CAAS. Location. Potato.  CAU.  Near Beijing.
  Tomato.  Treatments represent variation in irrigation source (tap
  water, waste water with different treatments), application method
  (furrow and subsoil drip), and strategy (FI, Deficit, and PRD).
  Nutrients not included (should be sufficient).  Different irrigation
  sources included for examining safety, not as nutrient source, but
  may nontheless have an effect.  Three replications.  Root density
  measured only for selected treatments, with no replications.
\item TODO: CAAS.  2006 and 2008 data.  2006 data bad (all
  measurements below DensRootTip).  Ridges.  Figure with field setup
  and measurement points.  Choose to include all (3D) data.  Table
  with plots, treatments, and yields.  FAOSTAT yield for reference.
\item TODO: CAU.  2007 and 2008 data, all good.  Ridges.  Figure with
  field setup and measurement points.  Only include 2D data. Table
  with plots, treatments, and yields.  FAOSTAT yield for reference.
\item TODO: ``Fixed'' model parameters.  Finding optimal free parameters
  with regard to $R^2$ from data.  \cite{nelder1965simplex}.
\item TODO: Find solution for both 1D and 2D model.  The 2D model is a
  porper superset of the 1D model (\ref{sec:mapping}) with one more
  parameter.  This means we can use a F-test to compare.
\end{itemize}

\begin{figure}[htbp]
  \input{sample-caas}
  \caption{Sampling points for CAAS site.  All distances in cm.  Root
    density has been sampled at the center of the ridge (X=0), at the
    center of the furrow (X=37.5), and halfway between the furrow and
    ridge (X=18.75).  This sampling was done both at the plant
    location (Y=0), and between the plants in the row (Y=15).  The
    sampling depth starts at soil surface and continues in 10 cm
    intervals to -70 cm.  There are no samples in the top of the
    ridge, above the original surface lavel.}
  \label{fig:sample-caas}
\end{figure}

\begin{figure}[htbp]
  \input{sample-cau}
  \caption{Sampling points for CAU site.  All distances in cm.  Root
    density has been sampled at at cross with sample and plant
    location, and at 10 and 20 cm at each side of the plant location
    within the ridge, and at 15 and 30 cm at each side across the
    ridge system.  The sampling depths start 15 centimeters above the
    original ground level within the ridge, and continues down to 55
    cm for at the plant location and 10 cm away, down to 45 cm 15 cm
    away, and down to 35 cm 20 and 30 cm away. We only use the data
    from the samplings across the ridge system and below the original
    ground level, marked $+$ on the figure.}
  \label{fig:sample-cau}
\end{figure}

\section{Results}

\begin{itemize}
\item TODO: 5 tabeller med resultater
\item TODO: Perhaps figure (or table) with results from one plot.
\end{itemize}

\begin{figure}[htbp]
  \input{compare-30}\\
  \input{compare-15}\\
  \input{compare-00}\\
  \input{compare+15}\\
  \input{compare+30}
  \caption{Estimated and observed root density for the 2008-07-24 CAU
    sampling for all treatments.  The different graphs represent
    distance to row.  The unit for the x-axis is cm/cm$^3$.}
  \label{fig:cau2008}
\end{figure}

No clear pattern emerge on the effect of the irrigation treatment on
the root zone.  The 2008 CAU data points toward a contraction of the
root zone for all treatments between the first and the second
sampling, see figure~\ref{fig:cau2008}.  This pattern is partly
supported by the 2006 CAU data.
\begin{figure}[htbp]
  \input{CAU2008a}\\
  \input{CAU2008b}
  \caption{Estimated root zone for the two sampling periods fo CAU
    2008.  Two interacting rows at x = 0 and 80 cm are shown.  The
    lines represent root density isoterms in cm/cm$^3$.}
  \label{fig:cau2008}
\end{figure}

\section{Discussion and concluding remarks}

We first presented a simple \twod{} extension of the parametric
\oned{} model found by \cite{gp74}, and shown how to calculate the
distribution parameters for the model from root mass, root zone depth,
and root zone width.

We then examined 28 datasets, plus five more that were created by
aggregating data from different plots.  The datasets spanned two
sites, two years, and two different sample dates within the same
growing season.  For the CAAS site we 8 \threed{} dataset, plus one
aggregated dataset.  For the CAU site we have 20 \twod{} datasets,
plus 4 aggregated dataset.  

The \twod{} model was a significant improvement over the \oned{} model
for all the aggregated datasets, half the \threed{} datasets, and 17
of the 20 \twod{} datasets.  The \twod{} model is never worse than the
\oned{} model.  For all the cases where we were unable to show a
statistically significant improvement for the \twod{} model, the best
fit for the root zone width was at least three times the distance
between rows, indicating a situation where there root zones of the
different plant rows were fully merged.  The $R^2$ values range from
0.73 to 0.96 for 27 of 28 datasets, with the last dataset having $R^2
= 0.30$.  For the aggregated datasets, $R^2$ range from 0.56 to 0.92.  

Based on this we believe the proposed model is useful for the intended
purpose of modelling water dynamics in row crops with PRD irrigation.

\section{Software and data availability}

Software to estimate root mass and root zone depth from 1D root
density data, and in addition root zone width from 2D root density
data, can be found at \url{http://www.daisy-model.org/} (look under
\texttt{Wiki}, \texttt{\textsc{gp2d}}).  Also supported is estimating
root density distribution from root mass, root zone depth, and
optionally, root zone width.  The program code is written in
\cplusplus{} and is covered by on open source license (\textsc{gnu
  lgpl}), so it can be freely incorporated in other models.

The root density data from the two field sites will be made available
at \url{http://www.safir4eu.org/} from TODO-XXXX-DATE
(look under \texttt{Trials}).

\section*{List of symbols}

\begin{tabular}{lll}
Symbol  & Unit    & Description\\\hline
$a$     & m$^{-1}$ & Root density distribution parameter\\
$a_z$   & m$^{-1}$ & Vertical root density distribution parameter\\
$a_x$   & m$^{-1}$ & Horizontal root density distribution parameter\\
$d_a$   & m       & Soil limited root depth\\
$d_c$   & m       & Crop potential root depth\\
$d_s$   & m       & Soil maximum root depth\\ 
$k^*$   &         & Soil root limit factor\\ 
$l_r$   & m/m$^2$ & Total root length per area\\ 
$l_R$   & m/m     & Total root length per length of row on one side\\ 
$L_0$   & m/m$^3$ & Average root density at soil surface\\
$L_{0,0}$& m/m$^3$ & Root density in row at soil surface\\
$L_m$   & m/m$^3$ & Minimal root density\\ 
$L_z$   & m/m$^3$ & Root density at soil depth $z$\\
$L_z^*$ & m/m$^3$ & Soil limited root density at soil depth $z$\\ 
$L_{z,x}$& m/m$^3$ & Root density at soil depth $z$ and distance $x$ from row\\
$L^*_{z,x}$& m/m$^3$ & Root density from multiple rows\\
$M_r$   & kg/m$^2$& Total root dry matter\\ 
$Q$     &         & Substitution variable\\
$R$     & m       & Distance between rows\\
$S_r$   & m/kg    & Specific root length\\ 
$W$     &         & Lambert-W function\\ 
$w_c$     & m     & Horizontal root radius\\
$x$     & m       & Horizontal distance from row \\
$z$     & m       & Soil depth \\
\end{tabular}

\bibliographystyle{elsart-harv}
\bibliography{../daisy}

\end{document}
