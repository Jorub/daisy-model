\newcommand{\figrorrende}[1]{\includegraphics[trim=9mm 0mm 14mm 12mm,clip]{fig/#1}}
\newcommand{\figrorrendel}[1]{\hspace*{-2cm}\figrorrende{#1}}

\chapter{2D plots}
\label{app:plot-2d}

In this appendix we present simulated 2D plots for water, bromide,
pendimethalin, and ioxynil.  The simulated values presented here are
all from the part of the soil with an aquitard bottom.  There are no
measurements to compare with, a major caveat for both the results and
discussion.  We use two kinds of graphs to capture the 2D structure.

The first kind depict static distribution in the soil.  Each graph has
horizontal distance from drain on the x-axis and height above surface
on the y-axis, using the same scale for both axes.  The graph
represents the computational soil area used in the simulation.
The right side is the center between two drains (8 meter for
R{\o}rrende), and the bottom is 2 meter below ground, where we use an
aquitard lower boundary with a calibrated aquifer.  The graphs are
color coded, where specific colors represent specific values for the
soil.  Each numeric cell in the computation has a color representing
the value within that cell.  Since cells are rectangular, the graphs
appear blocky.  The graphs are ordered chronologically, with the
earliest graph at the top left corner, continuing from there using the
normal reading direction.

The second kind of graph depicts horizontal or vertical movement.  For
the graphs depicting horizontal movement, the y-axis specifies height
above surface (negative number) and the x-axis movement away from
drain (usually also negative).  The horizontal movement at different
distances from the drain pipes is shown as separate plots on each
graph.  For the graphs depicting vertical movement, the axes are
swapped.  The individual plots represent different depths.  We use the
same flow units as we used for the original input, so e.g.\ pesticide
transport is given in g/ha.

\FloatBarrier
\section{Water}

Monthly snapshots of the soil water potential is shown on
figure~\ref{fig:Rorrende-pF} for all three drain seasons, and the
preceding summers. We see shallow groundwater at the beginning of the
first drain season, where we overestimate drain flow (see
figure~\ref{fig:season9899}).  The horizontal movement flux is largest
near the drain pipes, and at the bottom of the plow layer in
(figure~\ref{fig:Rorrende-water-horizontal}).  We also see the largest
upward flux below the drain pipes, and downward flux above the drain
pipes (figure~\ref{fig:Rorrende-water-vertical}).  Most of the
movement within the soil is through biopores
(figure~\ref{fig:Rorrende-water-biopore}).

\begin{figure}[htbp]\centering
  \begin{tabular}{ccc}
    \figrorrendel{Rorrende-pF-1998-5} & 
    \figrorrende{Rorrende-pF-1998-6} & 
    \figrorrende{Rorrende-pF-1998-7} \\
    \figrorrendel{Rorrende-pF-1998-8} & 
    \figrorrende{Rorrende-pF-1998-9} & 
    \figrorrende{Rorrende-pF-1998-10} \\
    \figrorrendel{Rorrende-pF-1998-11} & 
    \figrorrende{Rorrende-pF-1998-12} & 
    \figrorrende{Rorrende-pF-1999-1} \\
    \figrorrendel{Rorrende-pF-1999-2} & 
    \figrorrende{Rorrende-pF-1999-3} & 
    \figrorrende{Rorrende-pF-1999-4}\\
    \figrorrendel{Rorrende-pF-1999-5} & 
    \figrorrende{Rorrende-pF-1999-6} & 
    \figrorrende{Rorrende-pF-1999-7} \\
    \figrorrendel{Rorrende-pF-1999-8} & 
    \figrorrende{Rorrende-pF-1999-9} & 
    \figrorrende{Rorrende-pF-1999-10} \\
    \figrorrendel{Rorrende-pF-1999-11} & 
    \figrorrende{Rorrende-pF-1999-12} & 
    \figrorrende{Rorrende-pF-2000-1} \\
    \figrorrendel{Rorrende-pF-2000-2} & 
    \figrorrende{Rorrende-pF-2000-3} & 
    \figrorrende{Rorrende-pF-2000-4}\\
    \figrorrendel{Rorrende-pF-2000-5} & 
    \figrorrende{Rorrende-pF-2000-6} & 
    \figrorrende{Rorrende-pF-2000-7} \\
    \figrorrendel{Rorrende-pF-2000-8} & 
    \figrorrende{Rorrende-pF-2000-9} & 
    \figrorrende{Rorrende-pF-2000-10} \\
    \figrorrendel{Rorrende-pF-2000-11} & 
    \figrorrende{Rorrende-pF-2000-12} & 
    \figrorrende{Rorrende-pF-2001-1} 
  \end{tabular}
  
  \caption{Soil water pressure potential at the end of each month from
    May 1998 (top left) to January 2001 (bottom right).  The y-axis
    denotes depth, the x-axis distance from drain.  There are tick
    marks for every meter.  Blue denotes pF<0, white pF=1, yellow
    pF=2, orange pF=3, red pF=4, and black pF>5.}
\label{fig:Rorrende-pF}
\end{figure}

\begin{figure}[htbp]
  \centering
  \figtop{Rorrende-water-horizontal-1998}
  \figtop{Rorrende-water-horizontal-1999}
  \fig{Rorrende-water-horizontal-2000}
  
  \caption{Horizontal water flux between 1998-5-1 and 1999-5-1 (top),
    between 1999-5-1 and 2000-5-1 (center), and between 2000-5-1 and
    2001-2-1 (bottom).  The flux is shown on the x-axis (positive away
    from drain) as a function of depth shown on the y-axis.  The graph
    labels are the distance from drain in centimeters.}
  \label{fig:Rorrende-water-horizontal}
\end{figure}

\begin{figure}[htbp]
  \centering
  \figtop{Rorrende-water-1998}
  \fig{Rorrende-water-1999}
  \fig{Rorrende-water-2000}
  
  \caption{Total vertical water flux between 1998-5-1 and
    1999-5-1 (top), between 1999-5-1 and 2000-5-1 (center), and
    between 2000-5-1 and 2001-2-1 (bottom).  The flux is shown on the
    y-axis (positive up) as a function of distance from drain shown on
    the x-axis.  The graph labels are depths in centimeters above
    surface.}
  \label{fig:Rorrende-water-vertical}
\end{figure}

\begin{figure}[htbp]
  \centering
  \figtop{Rorrende-water-biopore-1998}
  \fig{Rorrende-water-biopore-1999}
  \fig{Rorrende-water-biopore-2000}
  
  \caption{Biopore water flux between 1998-5-1 and 1999-5-1 (top),
    between 1999-5-1 and 2000-5-1 (center), and between 2000-5-1 and
    2001-2-1 (bottom).  The flux is shown on the y-axis (positive up)
    as a function of distance from drain shown on the x-axis.  The
    graph labels are depths in centimeters above surface.}
  \label{fig:Rorrende-water-biopore}
\end{figure}

\FloatBarrier
\section{Bromide}

\begin{figure}[htbp]\centering
  \begin{tabular}{ccc}
    \figrorrendel{Rorrende-M-Bromide-1998-11} & 
    \figrorrende{Rorrende-M-Bromide-1998-12} & 
    \figrorrende{Rorrende-M-Bromide-1999-1} \\
    \figrorrendel{Rorrende-M-Bromide-1999-2} & 
    \figrorrende{Rorrende-M-Bromide-1999-3} & 
    \figrorrende{Rorrende-M-Bromide-1999-4} \\
    \figrorrendel{Rorrende-M-Bromide-1999-5} & 
    \figrorrende{Rorrende-M-Bromide-1999-6} & 
    \figrorrende{Rorrende-M-Bromide-1999-7} \\
    \figrorrendel{Rorrende-M-Bromide-1999-8} & 
    \figrorrende{Rorrende-M-Bromide-1999-9} & 
    \figrorrende{Rorrende-M-Bromide-1999-10} \\
    \figrorrendel{Rorrende-M-Bromide-1999-11} & 
    \figrorrende{Rorrende-M-Bromide-1999-12} & 
    \figrorrende{Rorrende-M-Bromide-2000-1} \\
    \figrorrendel{Rorrende-M-Bromide-2000-2} & 
    \figrorrende{Rorrende-M-Bromide-2000-3} & 
    \figrorrende{Rorrende-M-Bromide-2000-4}
  \end{tabular}
  
  \caption{Bromide soil content at the end of each month since first
    application in November 1998 (top left graph) until April 2000
    (bottom right graph).  The y-axis denotes depth, the x-axis
    distance from drain.  There are tick marks for every meter. The
    color scale is white<1 $\mu$g/l, yellow=10 $\mu$g/l, orange=100 $\mu$g/l,
    red=1 mg/l, and black>10 mg/l}
\label{fig:Rorrende-M-Bromide}
\end{figure}

\begin{figure}[htbp]
  \centering
  \fig{Rorrende-Bromide-horizontal-1998}
  \figtop{Rorrende-Bromide-1998}
  \fig{Rorrende-Bromide-biopore-1998}

  
  \caption{Bromide transport between 1998-5-1 and 1999-5-1.  The top
    graph show horizontal transport (top), the center graph show total
    vertical transport, and the bottom graph show biopore transport
    only.  The transport in the top graph is shown on the x-axis
    (positive away from drain) as a function of depth shown on the
    y-axis, with graph labels indicating the distance from drain in
    centimeters.  The transport on the two lower graphs are shown on
    the y-axis (positive up) as a function of distance from drain
    shown on the x-axis. The graph labels are depths in centimeters above
    surface.}
  \label{fig:Rorrende-Bromide-1998}
\end{figure}

\begin{figure}[htbp]
  \centering
  \fig{Rorrende-Bromide-horizontal-1999}
  \figtop{Rorrende-Bromide-1999}
  \fig{Rorrende-Bromide-biopore-1999}

  
  \caption{Bromide transport between 1999-5-1 and 2000-5-1.  The top
    graph show horizontal transport (top), the center graph show total
    vertical transport, and the bottom graph show biopore transport
    only.  The transport in the top graph is shown on the x-axis
    (positive away from drain) as a function of depth shown on the
    y-axis, with graph labels indicating the distance from drain in
    centimeters.  The transport on the two Lowery graphs are shown on
    the y-axis (positive up) as a function of distance from drain
    shown on the x-axis. The graph labels are depths in centimeters above
    surface.}
  \label{fig:Rorrende-Bromide-1999}
\end{figure}

\FloatBarrier
\section{Pendimethalin}

\begin{figure}[htbp]\centering
  \begin{tabular}{ccc}
    \figrorrendel{Rorrende-M-Pendimethalin-1999-11} & 
    \figrorrende{Rorrende-M-Pendimethalin-1999-12} & 
    \figrorrende{Rorrende-M-Pendimethalin-2000-1} \\
    \figrorrendel{Rorrende-M-Pendimethalin-2000-2} & 
    \figrorrende{Rorrende-M-Pendimethalin-2000-3} & 
    \figrorrende{Rorrende-M-Pendimethalin-2000-4} \\
    \figrorrendel{Rorrende-M-Pendimethalin-2000-5} & 
    \figrorrende{Rorrende-M-Pendimethalin-2000-6} & 
    \figrorrende{Rorrende-M-Pendimethalin-2000-7} \\
    \figrorrendel{Rorrende-M-Pendimethalin-2000-8} & 
    \figrorrende{Rorrende-M-Pendimethalin-2000-9} & 
    \figrorrende{Rorrende-M-Pendimethalin-2000-10} \\
    \figrorrendel{Rorrende-M-Pendimethalin-2000-11} & 
    \figrorrende{Rorrende-M-Pendimethalin-2000-12} & 
    \figrorrende{Rorrende-M-Pendimethalin-2001-1}
  \end{tabular}
  
  \caption{Pendimethalin soil content at the end of each month since
    first application in November 1999 (top left graph) until January
    2001 (bottom right graph).  The y-axis denotes depth, the x-axis
    distance from drain.  There are tick marks for every meter. The
    color scale is white<10 pg/l, yellow=1 ng/l, orange=0.1 $\mu$g/l,
    red=10 $\mu$g/l, and black>1 mg/l}
\label{fig:Rorrende-M-Pendimethalin}
\end{figure}

\begin{figure}[htbp]\centering
  \begin{tabular}{ccc}
    \figrorrendel{Rorrende-C-Pendimethalin-1999-11} & 
    \figrorrende{Rorrende-C-Pendimethalin-1999-12} & 
    \figrorrende{Rorrende-C-Pendimethalin-2000-1} \\
    \figrorrendel{Rorrende-C-Pendimethalin-2000-2} & 
    \figrorrende{Rorrende-C-Pendimethalin-2000-3} & 
    \figrorrende{Rorrende-C-Pendimethalin-2000-4} \\
    \figrorrendel{Rorrende-C-Pendimethalin-2000-5} & 
    \figrorrende{Rorrende-C-Pendimethalin-2000-6} & 
    \figrorrende{Rorrende-C-Pendimethalin-2000-7} \\
    \figrorrendel{Rorrende-C-Pendimethalin-2000-8} & 
    \figrorrende{Rorrende-C-Pendimethalin-2000-9} & 
    \figrorrende{Rorrende-C-Pendimethalin-2000-10} \\
    \figrorrendel{Rorrende-C-Pendimethalin-2000-11} & 
    \figrorrende{Rorrende-C-Pendimethalin-2000-12} & 
    \figrorrende{Rorrende-C-Pendimethalin-2001-1}
  \end{tabular}
  
  \caption{Pendimethalin soil water content at the end of each month
    since first application in November 1999 (top left graph) until
    January 2001 (bottom right graph).  The y-axis denotes depth, the
    x-axis distance from drain.  There are tick marks for every
    meter. The color scale is white<10 pg/l, yellow=1 ng/l, orange=0.1
    $\mu$g/l, red=10 $\mu$g/l, and black>1 mg/l}
\label{fig:Rorrende-C-Pendimethalin}
\end{figure}

\begin{figure}[htbp]
  \centering
  \fig{Rorrende-Pendimethalin-horizontal-1999}
  \figtop{Rorrende-Pendimethalin-1999}
  \fig{Rorrende-Pendimethalin-biopore-1999}
  
  \caption{Pendimethalin transport between 1999-5-1 and 2000-5-1.  The top
    graph show horizontal transport (top), the center graph show total
    vertical transport, and the bottom graph show biopore transport
    only.  The transport in the top graph is shown on the x-axis
    (positive away from drain) as a function of depth shown on the
    y-axis, with graph labels indicating the distance from drain in
    centimeters.  The transport on the two lower graphs are shown on
    the y-axis (positive up) as a function of distance from drain
    shown on the x-axis. The graph labels are depths in centimeters above
    surface.}
  \label{fig:Rorrende-Pendimethalin-1999}
\end{figure}

\begin{figure}[htbp]
  \centering
  \fig{Rorrende-Pendimethalin-horizontal-2000}
  \figtop{Rorrende-Pendimethalin-2000}
  \fig{Rorrende-Pendimethalin-biopore-2000}

  \caption{Pendimethalin transport between 2000-5-1 and 2001-2-1.  The top
    graph show horizontal transport (top), the center graph show total
    vertical transport, and the bottom graph show biopore transport
    only.  The transport in the top graph is shown on the x-axis
    (positive away from drain) as a function of depth shown on the
    y-axis, with graph labels indicating the distance from drain in
    centimeters.  The transport on the two lower graphs are shown on
    the y-axis (positive up) as a function of distance from drain
    shown on the x-axis. The graph labels are depths in centimeters above
    surface.}
  \label{fig:Rorrende-Pendimethalin-2000}
\end{figure}

\FloatBarrier
\section{Ioxynil}

\begin{figure}[htbp]\centering
  \begin{tabular}{ccc}
    \figrorrendel{Rorrende-M-Ioxynil-2000-11} & 
    \figrorrende{Rorrende-M-Ioxynil-2000-12} & 
    \figrorrende{Rorrende-M-Ioxynil-2001-1}
  \end{tabular}
  
  \caption{Ioxynil soil content at the end of each month since first
    application in November 2000 (top left graph) until January 2001
    (bottom right graph).  The y-axis denotes depth, the x-axis
    distance from drain.  There are tick marks for every meter. The
    color scale is white<10 pg/l, yellow=1 ng/l, orange=0.1 $\mu$g/l,
    red=10 $\mu$g/l, and black>1 mg/l}
\label{fig:Rorrende-M-Ioxynil}
\end{figure}

\begin{figure}[htbp]\centering
  \begin{tabular}{ccc}
    \figrorrendel{Rorrende-C-Ioxynil-2000-11} & 
    \figrorrende{Rorrende-C-Ioxynil-2000-12} & 
    \figrorrende{Rorrende-C-Ioxynil-2001-1}
  \end{tabular}
  
  \caption{Ioxynil soil water content at the end of each month since
    first application in November 2000 (left graph) until January 2001
    (right graph).  The y-axis denotes depth, the x-axis distance
    from drain.  There are tick marks for every meter. The color scale
    is white<10 pg/l, yellow=1 ng/l, orange=0.1 $\mu$g/l, red=10
    $\mu$g/l, and black>1 mg/l}
\label{fig:Rorrende-C-Ioxynil}
\end{figure}

\begin{figure}[htbp]
  \centering
  \fig{Rorrende-Ioxynil-horizontal-2000}
  \figtop{Rorrende-Ioxynil-2000}
  \fig{Rorrende-Ioxynil-biopore-2000}

  \caption{Ioxynil transport between 2000-11-1 and 2001-2-1.  The top
    graph show horizontal transport (top), the center graph show total
    vertical transport, and the bottom graph show biopore transport
    only.  The transport in the top graph is shown on the x-axis
    (positive away from drain) as a function of depth shown on the
    y-axis, with graph labels indicating the distance from drain in
    centimeters.  The transport on the two lower graphs are shown on
    the y-axis (positive up) as a function of distance from drain
    shown on the x-axis. The graph labels are depths in centimeters above
    surface.}
  \label{fig:Rorrende-Ioxynil-2000}
\end{figure}

%%% Local Variables: 
%%% TeX-master: "agrovand"
%%% End: 
