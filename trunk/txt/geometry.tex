\documentclass[a4paper,11pt,twoside]{article}
\usepackage{a4}
\usepackage[T1]{fontenc}
\usepackage[latin1]{inputenc}
\usepackage{hyperref}
\newcommand{\daisy}{{\sc daisy}}
\newcommand{\Daisy}{{\sc Daisy}}
\newcommand{\cplusplus}%
{{\leavevmode{\rm{\hbox{C\hskip -0.1ex\raise 0.5ex\hbox{\tiny ++}}}}}}
\newcommand{\Cplusplus}{\cplusplus}
\newcommand{\mshe}{Mike/\textsc{she}}
\newcommand{\wintel}{\texttt{win32}}
\newcommand{\dll}{\textsc{dll}}
\newcommand{\Dll}{\textsc{Dll}}
\newcommand{\gui}{\textsc{gui}}
\newcommand{\Gui}{\textsc{Gui}}
\newcommand{\unix}{Unix}
\newcommand{\dhi}{\textsc{dhi}}
\newcommand{\Dhi}{\textsc{Dhi}}
\newcommand{\api}{\textsc{api}}
\newcommand{\Api}{\textsc{Api}}
%\newcommand{\url}[1]{\linebreak[4]\texttt{<URL:#1>}}

%%% Local Variables: 
%%% mode: latex
%%% TeX-master: t
%%% End: 


\documentclass{article}

\begin{document}

\section*{The Geometry  of Daisy}

Many procesess in Daisy occur below the soil surface.  Now ``below
the soil surface'' is a rather large volume, which we can describe in
various levels of detail.  The \emph{geometry} concept covers how we
describe this volume.  In general, we divide the soil volume into a
finite number of smaller volumes (called \emph{cells}).  This process
is called \emph{discretization}.  We then pretent that the properties
of the soil volume, such as the water content, can be represented by a
single number for each cell.  This is unlike the physical models Daisy
is build on, which generally describe such properties as continuous
dstribution over the volume.  However, a continious distribution
would, in the general case, require infinite storage, and our
computers are only finite.  So the basic assumption behind a discrete
model like Daisy is that we can get close enough on the continious
distribution by using enough cells in our begin.

\section{Limiting factors}

There are several factors limiting the number of cells we can have in
our system description, the first we already mentioned: We need enough
storage for all the cells.  However, this is rarely the limiting
factor on modern computers, with plenty of memory.  Another more
philosophical problem is that of scale, physical models are usually
only applicable as a certain range, making the cells smaller than that
can make interpretation tricky.  For example, talking about water
potential makes little sense for cells smaller than the soil
particles.  However, the mathematical description often continues to
work fine even if after the physical interpretation break down, so
this is not really the limiting factor either.

\subsection{Big-O}

The real limiting factor tend to be computational complexity.
Computational complexity is traditionally described as a function of
the input size, using the Big-O notation.  For example, the turnover
processes is $O(n)$, where $n$ is the number of cells.  This means
that if you double the number of cells, the simulation will, at worst
(Big-O always describe the worst case), take twice as long to finish.
Note that we don't say anything about absolute simulation times, those
depend on the actual hardware you run the program on.

\subsection{Turnover}

Turnover is $O(n)$ because we treat each cell indvidually, without
regard to the other cells.  For transport we can obviously not ignore
the neighboring cells, as these are either the source or destination
for water or solute movement.  Or rather, we can with the simplest
models based on the concept of \emph{field capacity}.  Water move
downwards if the water content in the cell exceeds field capacity, and
stay put otherwise.  These models are $O(n)$ and very popular due to
their simplicity.  In Daisy we support one such model (with a bit
heuristics to add limited support for cabilary rise and avoid
oversaturated soil= for the situations where presice estimates of
water movement are not necessary.

\subsection{Transport}

The heuristic approach to transport described above gives poor
approximation when looking at relative short timesteps, or fine
grained spatial discretization.  A much better approach involves
noticing that the output from each cell to its neighbors depend on the
input it receives from its neighbors.  This kind of interdependence
can be written as a system of linear equations, which can be solved
using well known methods.  The problem is that, in the most general
case, the space need to store the equation system is the square of the
number of cells ($O(n^2)$, abusing the Big-O notation to express
memory requirements) , and the time complexity of the solution is
$O(n^4)$.  This means that if we double the number of cells, we need
four times as much memory, and we will have to wait eight times as
long for the answer.

\subsubsection{Iterations and time steps}

To make matters worse, the inputs and outputs of each cell also depend
on the hydraulic conductivity of the cell, which depends (in a
non-linear way) on the water content of the cell, which we only know
after we have solved the equation system.  Which means we need to
iterate over the system until we found a solution.  If the system is
very dynamic (water content changes fast), this iteration process
might not be enough, and we need to use smaller time steps in order to
find a solution.  None of these factors have a simple dependency on
the number of cells, so they don't affect the Big-O numbers, but they
affect the real time used, further 

\subsubsection{Our salvation}

Luckily, we can usually do better than the worst case.  Each cell
really only depend on the cells in its neighborhood, which mean that
we can ignore dependencies on distant cells.  Using this fact we can
reduce both the memory requirement and the time complexity.  How much
we can reduce them depend on the specific geometry we use, but if we
for example allow us self to ignore all lateral movement, both can be
reduced to $O(n)$.  Due to the need for iterations and time step
reductions, the solution is still much slower than with the the simple
heuristic models, but at least the problem does no longer grow with
the cube of the number of cells.  This leads us to a discussion of the
number of spatial dimensions we have to take into account in our
geometry.

\section{Dimensions}

One way to reduce the number of cells needed is to assume the system
is homogeneous along some spatial dimensions.  Reducing the number of
dimensions drastically reduce the number of cells.  If we assume the
same spatial resolution ans size along each dimension the number of
cells needed is $s^d$, where $s$ is the resulution (number of
intervals) along one dimension, and $d$ is the number of dimensions.
For example, imagine a 1m $\times$ 1m box, with a resolution of 1cm.
This gives us 100 intervals per dimension, and the cells needed is
$100^1 = 100$ in a one dimension geometry, $100^2 = 10,000$ in a two
dimensional geometry, and $100^3 = 1,000,000$ in a three dimensional
geometry.  In a zero dimensional geometry the cells needed is just
$100^0 = 1$, so we will look at that case first.

\subsection{Zero dimensions}

In the simplest case, we assume the soil is fully homogenious.  This
asumption makes sense for a field with a dense root network, that is
able to catch all the water added to system from above, and with deep
groundwater so no water is added from below (cabilary rise).  For many
purposes, such as e.g. irrigation control, or modelling of growth of
non-stresses crops, such a zero dimensional model is a good match.  It
is, however, not supported by Daisy, as the original focus of that
model was leaching, for which at least one dimension is needed.

\subsection{One dimension}

Water and solutes take some time moving down the soil profile, during
which it may be affected by root zone processes such as crop uptake.
This means that if we want to model percolation or leaching, we need
at least one dimension to describe vertical movement.

A one dimensional model is a good fit for a densily distributed crop
(such as wheat) on flat field, and can also be used on such a field
before the root system is fully devloped, as it can model the physical
differences in the soil quality that always exist between the plowing
layer, the rest of the root zone, and the soil below the root zone.
The soil below the root zone is important to take into account if
there is capilary rise.

\subsection{Two dimensions}

If we want to model the system on a scale smaller than the distance
between the crops, we need additional dimensions.  For row crops,
including ridged soil, we can sometimes get away with assuming the
field is homogeneous along the rows, so we only use one additional
dimension to model the heterogeneity across the rows.  Other
situations we might want to model in two dimensions is a field with
drain pipes, where we will asuume the field is homegenious along the
drain pipes, and a field with a height gradient.

\subsection{Three dimensions}

We need to go to a full three dimension model for a field where more
than one of the above mentioned factors must be modelled, and they are
not parallel.  We might also need three dimensions for a heterogenious
soil, if the soil physics is so we cannot ignore lateral movement.  

If we want to model water distribution from a drip point with subsoil
irrigation, we will need three dimension using Caretesian coordinates,
but only two dimensions if we let the second dimension be the
lateral distance from the drip point.

At this point, Daisy is structurally prepared for three coordinates,
but does not support it yet.  There are no plans to support Polar
coordinates.

\subsection{Axes}

By convention, we let \emph{z} denote the vertical axis, and put the
\emph{x} and \emph{y} axes on the horizontal plane.  The axes are
always ortogonal to each other.  The vertical axes is shared between
all geometries, and always (unlike common practice) written first,
followed by \emph{x} (for two and three dimensional geomentries) and
\emph{y} (for three dimensional geometries).  On the z axis, we by
convention use zero for the soil surface, and negative numbers to
denote subsoil depths.  We talk about \texttt{top} and \texttt{bottom}
for higher and lower number of the z axes.  For the x axis, the \emph{left}
side is be convention zero, and it grows positive towards the
\emph{right} side of the volume.  For the y axis, the \emph{front} is
by convention zero, and it grows positive towards the \emph{back} of
the volume.

\section{Cell shapes }

We want to cover the relevant subsoil volume with non-overlapping
cells, which makes some limitations on what shapes we can use for the
cells.  

\subsection{Concerns}

Some more concerns are listed here.

\paragraph{User interface}
Most users of Daisy will prefer not having to deal with the
discretization of the volume, and rely on whatever heuritics Daisy can
provide for the job.  However, for those who do care, including the
authors of the program, we would like cell shapes that makes it easy
to specify the grid.

\paragraph{Flexibility}
We would like small cells near ``interesing'' places in the soil,
where we can expect high pressure gradients.  This would typically be
the top of the soil, and cells near drain pipes.  In the more mundane
parts of the soil, where things happens more slowly, we would like
larger cells to keep the total number of cells down.

\paragraph{Flow concerns}
For a good numeric solution to the flow between cells, we would like
them shaped so the content of the cell is a good indicator of the flow
through the faces connecting the cell and each of its neighbors.  That
means we would like all parts of the faces to be approximately the
same distance to the cell center, or in other words, we would like the
cells to be approximately spherical.  Of course we can't use spheres,
as we can't in general divide a volume into non-overlapping spheres.

\paragraph{Gravity}
If the cells have two vertical faces, we can calculate on a purely
gravity based flux, ignoring horizontal movement.  Such a transport
model can be good to have a reserve, if the more general models fail
to provide a solution.

\subsection{Boxes}

In the primary discretization used in Daisy, the cells are organized
in rectangular grid, with the grid planes parallel to two axes.  This
means the cells will be rectangular boxes, each with four vertical
faces, and two horizontal faces.  This is easy to specify, just a list
of numbers for each axis representing where the grid planes cross it.
\begin{verbatim}
  (zplus -1 -2 -5 -10 -20 -30 -50 -70 -90 -100 [cm])
  (xplus 10 20 30 40 50 60 70 80 90 100 [cm])
  (yplus 10 20 30 40 50 60 70 80 90 100 [cm])
\end{verbatim}
Here we divided one cubic meter of volume into 1000 cells.  The left
and right faces of the cells are in the particular discretiation
always 10 cm apart, and so are the front and back faces of the cells.
We do make use of the flexibility in the z axis, by making the top and
bottom face of the cells closer to each other for the top cells, and
farther apart for the cells near the bottom.

\subsection{Other shapes}

The transport algorithms are written so this can easily be extended to
trapez shaped cells, where four of the walls in the cells are still
vertical, but the floor and ceiling may have a slope.  The primary
transport algorithm is actuelly written so it will work with any
polygons (3D?), with the already named restrictions that it works
better the closer the cells are to spheres.  However, the simpler
reserve algorithms does require vertical columns of cells for
gravitational flow.

\section{Beyond cells}

Thenumber and volume of the cells are the basis of the
discretizations, but various models needs more detailed information
about geometry, which we will describe here.

\paragraph{Turnover and depth}
In general, the turnover functions don't need any geometry information
beyond the number of cells, not even the size of the cells as Daisy
will store concentratiuons rather than absolute content.  A few
turnover functions are affected by depth, so they will be able to ask
the geometry for the depth of the center of the node.

\paragraph{Projection on z-axes for 1D surface interaction}
Daisy allows you to combine one dimensional models for ``surface''
processes, such as crops and management, with a two or three
dimensional description of the soil.  For this to work, the geometry
module will be able to project values into one dimensional intervals
on the z axis.

\paragraph{Faces and face vectors}
For transport, we need an additional abstraction, namely the
\emph{face} that connects two neighbor cells.  To calculate diffusion
flow between two cells, we need the area of the face, and the length
of the \emph{face vector} connecting the two cell centers.  For
gravitational flow, we additionally need the difference in height of
the two cell centers.  That happens to be the length of the face
vector, multiplied with sinus to the angle of the vector cmpared to
the horizontal plane.  Daisy will keep information about flow through
each face, positive numbers indicate that the flow follow the
direction of the face vector, and vice versa.

\paragraph{Logging user specified volume}
Daisy allows logging content in, and flow in and out of, a user
specified volume.  To support this, the geometry module has functions
to find the toal content of the cells that overlap such a volume (for
partical overlaps, only the fraction overlapping the volume is
counted), as well as the total flow for each face vector that cross
the surface of the volume.  Currently only box volumes are supported,
with box edges that must be parallel to the axes.  With these, you can
log flow through each of the six faces of the box seperately.

\paragraph{Plotting}

For visualizing the geometry, information about \emph{edge} and
\emph{corners} may be useful.  An edge is the line connecting two or
more faces, and corners are the points connecting edges.

\section{API}

\section{Transport solution}

The water, solute, and heat transport models all relies on solving
linear equation systems which can be rewritten to the form $A x = b$,
where $x$ is a vector of size $n$ (the number of cells) and represents
our solution in the form of water pressure $h$, solute concentration
$C$, or temperature $T$.  $A$ is a $n\cross{}n$ matrix, where each row
represents the solution for a specific cell, and each column
represents the influence from a specific cell on the other cells.  So
the value of row 3, column six represents the influence of the state
of cell 6 of the solution for cell 3.  $b$ is a vector of size $n$,
which represents solution independent influences on the solution for
each sell.  That is, the old value for the cell, and source/sink terms
that are independent of the solution.

\end{document}
