\newcommand{\figestrup}[1]{\includegraphics[trim=12mm 0mm 17mm 9mm,clip]{fig/#1}}

\subsection*{Estrup 2D}

\begin{figure}[htbp]\centering
  \begin{tabular}{ccc}
    \figestrup{Estrup-pF-2000-5} & 
    \figestrup{Estrup-pF-2000-6} & 
    \figestrup{Estrup-pF-2000-7} \\
    \figestrup{Estrup-pF-2000-8} & 
    \figestrup{Estrup-pF-2000-9} & 
    \figestrup{Estrup-pF-2000-10} \\
    \figestrup{Estrup-pF-2000-11} & 
    \figestrup{Estrup-pF-2000-12} & 
    \figestrup{Estrup-pF-2001-1} \\
    \figestrup{Estrup-pF-2001-2} & 
    \figestrup{Estrup-pF-2001-3} & 
    \figestrup{Estrup-pF-2001-4}
  \end{tabular}
  
  \caption{Estrup soil potential at the end of each month since first
    application of Bromide.  The z-axis denotes depth, the x-axis
    distance from drain.  There are tick marks for every meter.  Blue
    denotes pF<0, white pF=1, yellow pF=2, orange pF=3, red pF=4, and
    black pF>5.}
\label{fig:Estrup-pF-2000}
\end{figure}\FloatBarrier

\begin{figure}[htbp]\centering
  \begin{tabular}{ccc}
    \figestrup{Estrup-pF-2001-5} & 
    \figestrup{Estrup-pF-2001-6} & 
    \figestrup{Estrup-pF-2001-7} \\
    \figestrup{Estrup-pF-2001-8} & 
    \figestrup{Estrup-pF-2001-9} & 
    \figestrup{Estrup-pF-2001-10} \\
    \figestrup{Estrup-pF-2001-11} & 
    \figestrup{Estrup-pF-2001-12} & 
    \figestrup{Estrup-pF-2002-1} \\
    \figestrup{Estrup-pF-2002-2} & 
    \figestrup{Estrup-pF-2002-3} & 
    \figestrup{Estrup-pF-2002-4}
  \end{tabular}
  
  \caption{Estrup soil potential at the end of each month second year
    after application of Bromide.  The z-axis denotes depth, the
    x-axis distance from drain.  There are tick marks for every meter.
    Blue denotes pF<0, white pF=1, yellow pF=2, orange pF=3, red pF=4,
    and black pF>5.}
\label{fig:Estrup-pF-2001}
\end{figure}\FloatBarrier

\begin{figure}[htbp]
  \centering
  \includegraphics{fig/Estrup-water-horizontal-2000}
  
  \caption{Estrup total horizontal water flux between 2000-5-1 and
    2001-5-1.  The flux is shown on the x-axis (positive away from
    drain) as a function of depth shown on the y-axis.  The graph
    labels are the distance from drain in centimeters.}
  \label{fig:Estrup-water-horizontal-2000}
\end{figure}\FloatBarrier

\begin{figure}[htbp]
  \centering
  \includegraphics{fig/Estrup-water-2000}
  
  \caption{Estrup total vertical water flux between 2000-5-1 and
    2001-5-1.  The flux is shown on the y-axis (positive up) as a
    function of distance from drain shown on the y-axis.  The graph
    labels are depths in centimeters above surface.}
  \label{fig:Estrup-water-2000}
\end{figure}\FloatBarrier

\begin{figure}[htbp]
  \centering
  \includegraphics{fig/Estrup-water-biopore-2000}
  
  \caption{Estrup total biopore water flux between 2000-5-1 and
    2001-5-1.  The flux is shown on the y-axis (positive up) as a
    function of distance from drain shown on the y-axis.  The graph
    labels are depths in centimeters above surface.}
  \label{fig:Estrup-water-biopore-2000}
\end{figure}\FloatBarrier

\begin{figure}[htbp]
  \centering
  \includegraphics{fig/Estrup-water-horizontal-2001}
  
  \caption{Estrup total horizontal water flux between 2001-5-1 and
    2002-5-1.  The flux is shown on the x-axis (positive away from
    drain) as a function of depth shown on the y-axis.  The graph
    labels are the distance from drain in centimeters.}
  \label{fig:Estrup-water-2001-horizontal}
\end{figure}\FloatBarrier

\begin{figure}[htbp]
  \centering
  \includegraphics{fig/Estrup-water-2001}
  
  \caption{Estrup total vertical water flux between 2001-5-1 and
    2002-5-1.  The flux is shown on the y-axis (positive up) as a
    function of distance from drain shown on the y-axis.  The graph
    labels are depths in centimeters above surface.}
  \label{fig:Estrup-water-2001}
\end{figure}\FloatBarrier

\begin{figure}[htbp]
  \centering
  \includegraphics{fig/Estrup-water-biopore-2001}
  
  \caption{Estrup total biopore water flux between 2001-5-1 and
    2002-5-1.  The flux is shown on the y-axis (positive up) as a
    function of distance from drain shown on the y-axis.  The graph
    labels are depths in centimeters above surface.}
  \label{fig:Estrup-water-biopore-2001}
\end{figure}\FloatBarrier

\begin{figure}[htbp]\centering
  \begin{tabular}{ccc}
    \figestrup{Estrup-M-Bromide-2000-5} & 
    \figestrup{Estrup-M-Bromide-2000-6} & 
    \figestrup{Estrup-M-Bromide-2000-7} \\
    \figestrup{Estrup-M-Bromide-2000-8} & 
    \figestrup{Estrup-M-Bromide-2000-9} & 
    \figestrup{Estrup-M-Bromide-2000-10} \\
    \figestrup{Estrup-M-Bromide-2000-11} & 
    \figestrup{Estrup-M-Bromide-2000-12} & 
    \figestrup{Estrup-M-Bromide-2001-1} \\
    \figestrup{Estrup-M-Bromide-2001-2} & 
    \figestrup{Estrup-M-Bromide-2001-3} & 
    \figestrup{Estrup-M-Bromide-2001-4}
  \end{tabular}
  
  \caption{Estrup Bromide soil content at the end of each month since
    first application of Bromide.  The z-axis denotes depth, the x-axis distance from drain.  There are tick marks for every
    meter. The color scale is white<10 pg/l, yellow=1 ng/l,
    orange=0.1 $\mu$g/l, red=10 $\mu$g/l, and black>1 mg/l}
\label{fig:Estrup-Bromide-2000}
\end{figure}\FloatBarrier

\begin{figure}[htbp]\centering
  \begin{tabular}{ccc}
    \figestrup{Estrup-M-Bromide-2001-5} & 
    \figestrup{Estrup-M-Bromide-2001-6} & 
    \figestrup{Estrup-M-Bromide-2001-7} \\
    \figestrup{Estrup-M-Bromide-2001-8} & 
    \figestrup{Estrup-M-Bromide-2001-9} & 
    \figestrup{Estrup-M-Bromide-2001-10} \\
    \figestrup{Estrup-M-Bromide-2001-11} & 
    \figestrup{Estrup-M-Bromide-2001-12} & 
    \figestrup{Estrup-M-Bromide-2002-1} \\
    \figestrup{Estrup-M-Bromide-2002-2} & 
    \figestrup{Estrup-M-Bromide-2002-3} & 
    \figestrup{Estrup-M-Bromide-2002-4}
  \end{tabular}
  
  \caption{Estrup Bromide soil content at the end of each month second
    year after application of Bromide.  The z-axis denotes depth, the
    x-axis distance from drain.  There are tick marks for every
    meter. The color scale is white<10 pg/l, yellow=1 ng/l, orange=0.1
    $\mu$g/l, red=10 $\mu$g/l, and black>1 mg/l}
\label{fig:Estrup-Bromide-2001}
\end{figure}\FloatBarrier

\begin{figure}[htbp]
  \centering
  \includegraphics{fig/Estrup-Bromide-horizontal-2000}
  
  \caption{Estrup total horizontal Bromide flow between 2000-5-1 and
    2001-5-1.  The flow is shown on the x-axis (positive away from
    drain) as a function of depth shown on the y-axis.  The graph
    labels are the distance from drain in centimeters.}
  \label{fig:Estrup-Bromide-2000-horizontal}
\end{figure}\FloatBarrier

\begin{figure}[htbp]
  \centering
  \includegraphics{fig/Estrup-Bromide-2000}
  
  \caption{Estrup total vertical Bromide flow between 2000-5-1 and
    2001-5-1.  The flow is shown on the y-axis (positive up) as a
    function of distance from drain shown on the y-axis.  The graph
    labels are depths in centimeters above surface.}
  \label{fig:Estrup-Bromide-2000-vertical}
\end{figure}\FloatBarrier

\begin{figure}[htbp]
  \centering
  \includegraphics{fig/Estrup-Bromide-biopore-2000}
  
  \caption{Estrup total biopore Bromide flow between 2000-5-1 and
    2001-5-1.  The flow is shown on the y-axis (positive up) as a
    function of distance from drain shown on the y-axis.  The graph
    labels are depths in centimeters above surface.}
  \label{fig:Estrup-Bromide-biopore-2000}
\end{figure}\FloatBarrier

\begin{figure}[htbp]
  \centering
  \includegraphics{fig/Estrup-Bromide-horizontal-2001}
  
  \caption{Estrup total horizontal Bromide flow between 2001-5-1 and
    2002-5-1.  The flow is shown on the x-axis (positive away from
    drain) as a function of depth shown on the y-axis.  The graph
    labels are the distance from drain in centimeters.}
  \label{fig:Estrup-Bromide-2001-horizontal}
\end{figure}\FloatBarrier

\begin{figure}[htbp]
  \centering
  \includegraphics{fig/Estrup-Bromide-2001}
  
  \caption{Estrup total vertical Bromide flow between 2001-5-1 and
    2002-5-1.  The flow is shown on the y-axis (positive up) as a
    function of distance from drain shown on the y-axis.  The graph
    labels are depths in centimeters above surface.}
  \label{fig:Estrup-Bromide-2001-vertical}
\end{figure}\FloatBarrier

\begin{figure}[htbp]
  \centering
  \includegraphics{fig/Estrup-Bromide-biopore-2001}
  
  \caption{Estrup total biopore Bromide flow between 2001-5-1 and
    2002-5-1.  The flow is shown on the y-axis (positive up) as a
    function of distance from drain shown on the y-axis.  The graph
    labels are depths in centimeters above surface.}
  \label{fig:Estrup-Bromide-biopore-2001}
\end{figure}\FloatBarrier

\begin{figure}[htbp]\centering
  \begin{tabular}{ccc}
    \figestrup{Estrup-M-Glyphosate-2000-5} & 
    \figestrup{Estrup-M-Glyphosate-2000-6} & 
    \figestrup{Estrup-M-Glyphosate-2000-7} \\
    \figestrup{Estrup-M-Glyphosate-2000-8} & 
    \figestrup{Estrup-M-Glyphosate-2000-9} & 
    \figestrup{Estrup-M-Glyphosate-2000-10} \\
    \figestrup{Estrup-M-Glyphosate-2000-11} & 
    \figestrup{Estrup-M-Glyphosate-2000-12} & 
    \figestrup{Estrup-M-Glyphosate-2001-1} \\
    \figestrup{Estrup-M-Glyphosate-2001-2} & 
    \figestrup{Estrup-M-Glyphosate-2001-3} & 
    \figestrup{Estrup-M-Glyphosate-2001-4}
  \end{tabular}
  
  \caption{Estrup Glyphosate soil content at the end of each month
    since first application of Bromide.  The z-axis denotes depth, the
    x-axis distance from drain.  There are tick marks for every
    meter. The color scale is white<10 pg/l, yellow=1 ng/l, orange=0.1
    $\mu$g/l, red=10 $\mu$g/l, and black>1 mg/l}
\label{fig:Estrup-M-Glyphosate-2000}
\end{figure}\FloatBarrier

\begin{figure}[htbp]\centering
  \begin{tabular}{ccc}
    \figestrup{Estrup-C-Glyphosate-2000-5} & 
    \figestrup{Estrup-C-Glyphosate-2000-6} & 
    \figestrup{Estrup-C-Glyphosate-2000-7} \\
    \figestrup{Estrup-C-Glyphosate-2000-8} & 
    \figestrup{Estrup-C-Glyphosate-2000-9} & 
    \figestrup{Estrup-C-Glyphosate-2000-10} \\
    \figestrup{Estrup-C-Glyphosate-2000-11} & 
    \figestrup{Estrup-C-Glyphosate-2000-12} & 
    \figestrup{Estrup-C-Glyphosate-2001-1} \\
    \figestrup{Estrup-C-Glyphosate-2001-2} & 
    \figestrup{Estrup-C-Glyphosate-2001-3} & 
    \figestrup{Estrup-C-Glyphosate-2001-4}
  \end{tabular}
  
  \caption{Estrup Glyphosate soil water concentration at the end of
    each month since first application of Bromide.  The z-axis denotes
    depth, the x-axis distance from drain.  There are tick marks for
    every meter. The color scale is white<10 pg/l, yellow=1 ng/l, orange=0.1
    $\mu$g/l, red=10 $\mu$g/l, and black>1 mg/l}
\label{fig:Estrup-C-Glyphosate-2000}
\end{figure}\FloatBarrier

\begin{figure}[htbp]
  \centering
  \includegraphics{fig/Estrup-Glyphosate-horizontal-2000}
  
  \caption{Estrup total horizontal Glyphosate flow between 2000-5-1 and
    2001-5-1.  The flow is shown on the x-axis (positive away from
    drain) as a function of depth shown on the y-axis.  The graph
    labels are the distance from drain in centimeters.}
  \label{fig:Estrup-Glyphosate-2000-horizontal}
\end{figure}\FloatBarrier

\begin{figure}[htbp]
  \centering
  \includegraphics{fig/Estrup-Glyphosate-2000}
  
  \caption{Estrup total vertical Glyphosate flow between 2000-5-1 and
    2001-5-1.  The flow is shown on the y-axis (positive up) as a
    function of distance from drain shown on the y-axis.  The graph
    labels are depths in centimeters above surface.}
  \label{fig:Estrup-Glyphosate-2000}
\end{figure}\FloatBarrier

\begin{figure}[htbp]
  \centering
  \includegraphics{fig/Estrup-Glyphosate-biopore-2000}
  
  \caption{Estrup total biopore Glyphosate flow between 2000-5-1 and
    2001-5-1.  The flow is shown on the y-axis (positive up) as a
    function of distance from drain shown on the y-axis.  The graph
    labels are depths in centimeters above surface.}
  \label{fig:Estrup-Glyphosate-biopore-2000}
\end{figure}\FloatBarrier

