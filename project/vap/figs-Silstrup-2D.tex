\newcommand{\figsilstrupl}[1]{\figl\includegraphics[trim=8mm 0mm 12mm 7mm,clip]{fig/#1}}
\newcommand{\figsilstrup}[1]{\includegraphics[trim=8mm 0mm 12mm 7mm,clip]{fig/#1}}
\newcommand{\fluxtop}[1]{\figl\includegraphics[trim=0mm 10mm 0mm 0mm,clip]{fig/#1}}
\chapter{Silstrup 2D dynamics}
\label{app:silstrup-2d}

In this appendix the simulated 2D dynamics for water, bromide and
pesticides of the Silstrup site is presented.  There are no
measurements to compare with.

The block diagrams
(figure~\ref{fig:Silstrup-pF-2000},~\ref{fig:Silstrup-pF-2001},~\ref{fig:Silstrup-Bromide-2000},~\ref{fig:Silstrup-Bromide-2001},~\ref{fig:Silstrup-M-Metamitron-2000},~\ref{fig:Silstrup-C-Metamitron-2000},~\ref{fig:Silstrup-M-Glyphosate-2001},
and~\ref{fig:Silstrup-C-Glyphosate-2001}) all share the same format.
Each graph has horizontal distance from drain on the x-axis, and
height above surface on the y-axis.  The graph represent the the
computational soil area used in the simulation.  The right side is the
center between two drains, and the bottom is 5 meter, where we use the
measured groundwater pressure table as the lower boundary.  The graphs
are color coded, where specific colors represent specific values for
the soil, at the end of the month indicated by the graph title.  Each
numeric cell in the computation has a color representing the value
within that cell.  Since cells are rectangular, the graphs appear
blocky.  


\begin{figure}[htbp]\centering
  \begin{tabular}{ccc}
    \figsilstrupl{Silstrup-pF-2000-5} & 
    \figsilstrup{Silstrup-pF-2000-6} & 
    \figsilstrup{Silstrup-pF-2000-7} \\
    \figsilstrupl{Silstrup-pF-2000-8} & 
    \figsilstrup{Silstrup-pF-2000-9} & 
    \figsilstrup{Silstrup-pF-2000-10} \\
    \figsilstrupl{Silstrup-pF-2000-11} & 
    \figsilstrup{Silstrup-pF-2000-12} & 
    \figsilstrup{Silstrup-pF-2001-1} \\
    \figsilstrupl{Silstrup-pF-2001-2} & 
    \figsilstrup{Silstrup-pF-2001-3} & 
    \figsilstrup{Silstrup-pF-2001-4}
  \end{tabular}
  
  \caption{Silstrup soil potential at the end of each month since
    first application of bromide.  The y-axis denotes depth, the
    x-axis distance from drain.  There are tick marks for every meter.
    Blue denotes pF<0, white pF=1, yellow pF=2, orange pF=3, red pF=4,
    and black pF>5.}
\label{fig:Silstrup-pF-2000}
\end{figure}\FloatBarrier

\begin{figure}[htbp]\centering
  \begin{tabular}{ccc}
    \figsilstrupl{Silstrup-pF-2001-5} & 
    \figsilstrup{Silstrup-pF-2001-6} & 
    \figsilstrup{Silstrup-pF-2001-7} \\
    \figsilstrupl{Silstrup-pF-2001-8} & 
    \figsilstrup{Silstrup-pF-2001-9} & 
    \figsilstrup{Silstrup-pF-2001-10} \\
    \figsilstrupl{Silstrup-pF-2001-11} & 
    \figsilstrup{Silstrup-pF-2001-12} & 
    \figsilstrup{Silstrup-pF-2002-1} \\
    \figsilstrupl{Silstrup-pF-2002-2} & &
  \end{tabular}
  
  \caption{Silstrup soil potential at the end of each month second year
    after application of bromide.  The y-axis denotes depth, the
    x-axis distance from drain.  There are tick marks for every meter.
    Blue denotes pF<0, white pF=1, yellow pF=2, orange pF=3, red pF=4,
    and black pF>5.}
\label{fig:Silstrup-pF-2001}
\end{figure}\FloatBarrier

\begin{figure}[htbp]
  \centering
  \figtop{Silstrup-water-horizontal-2000}
  \fig{Silstrup-water-horizontal-2001}
  
  \caption{Silstrup total horizontal water flux between 2000-5-1 and
    2001-5-1 (top) and between 2001-5-1 and 2002-3-1 (bottom).  The
    flux is shown on the x-axis (positive away from drain) as a
    function of depth shown on the y-axis.  The graph labels are the
    distance from drain in centimeters.}
  \label{fig:Silstrup-water-horizontal}
\end{figure}\FloatBarrier

\begin{figure}[htbp]
  \centering
  \figtop{Silstrup-water-2000}
  \fig{Silstrup-water-biopore-2000}
  
  \caption{Silstrup vertical water flux between 2000-5-1 and
    2001-5-1.  Top graph show total flux, bottom graph only biopores.  The flux is shown on the y-axis (positive up) as a
    function of distance from drain shown on the x-axis.  The graph
    labels are depths in centimeters above surface.}
  \label{fig:Silstrup-water-2000}
\end{figure}\FloatBarrier

\begin{figure}[htbp]
  \centering
  \figtop{Silstrup-water-2001}
  \fig{Silstrup-water-biopore-2001}
  
  \caption{Silstrup vertical water flux between 2001-5-1 and 2002-3-1.
    Top graph show total flux, bottom graph only biopores.  The flux
    is shown on the y-axis (positive up) as a function of distance
    from drain shown on the x-axis.  The graph labels are depths in
    centimeters above surface.}
  \label{fig:Silstrup-water-2001}
\end{figure}\FloatBarrier

\begin{figure}[htbp]\centering
  \begin{tabular}{ccc}
    \figsilstrupl{Silstrup-M-Bromide-2000-5} & 
    \figsilstrup{Silstrup-M-Bromide-2000-6} & 
    \figsilstrup{Silstrup-M-Bromide-2000-7} \\
    \figsilstrupl{Silstrup-M-Bromide-2000-8} & 
    \figsilstrup{Silstrup-M-Bromide-2000-9} & 
    \figsilstrup{Silstrup-M-Bromide-2000-10} \\
    \figsilstrupl{Silstrup-M-Bromide-2000-11} & 
    \figsilstrup{Silstrup-M-Bromide-2000-12} & 
    \figsilstrup{Silstrup-M-Bromide-2001-1} \\
    \figsilstrupl{Silstrup-M-Bromide-2001-2} & 
    \figsilstrup{Silstrup-M-Bromide-2001-3} & 
    \figsilstrup{Silstrup-M-Bromide-2001-4}
  \end{tabular}
  
  \caption{Silstrup bromide soil content at the end of each month
    since first application of bromide.  The y-axis denotes depth, the
    x-axis distance from drain.  There are tick marks for every
    meter. The color scale is white<10 pg/l, yellow=1 ng/l, orange=0.1
    $\mu$g/l, red=10 $\mu$g/l, and black>1 mg/l}
\label{fig:Silstrup-Bromide-2000}
\end{figure}\FloatBarrier

\begin{figure}[htbp]\centering
  \begin{tabular}{ccc}
    \figsilstrupl{Silstrup-M-Bromide-2001-5} & 
    \figsilstrup{Silstrup-M-Bromide-2001-6} & 
    \figsilstrup{Silstrup-M-Bromide-2001-7} \\
    \figsilstrupl{Silstrup-M-Bromide-2001-8} & 
    \figsilstrup{Silstrup-M-Bromide-2001-9} & 
    \figsilstrup{Silstrup-M-Bromide-2001-10} \\
    \figsilstrupl{Silstrup-M-Bromide-2001-11} & 
    \figsilstrup{Silstrup-M-Bromide-2001-12} & 
    \figsilstrup{Silstrup-M-Bromide-2002-1} \\
    \figsilstrupl{Silstrup-M-Bromide-2002-2} &  & 
  \end{tabular}
  
  \caption{Silstrup bromide soil content at the end of each month
    second year after application of bromide.  The y-axis denotes
    depth, the x-axis distance from drain.  There are tick marks for
    every meter. The color scale is white<10 pg/l, yellow=1 ng/l,
    orange=0.1 $\mu$g/l, red=10 $\mu$g/l, and black>1 mg/l}
\label{fig:Silstrup-Bromide-2001}
\end{figure}\FloatBarrier

\begin{figure}[htbp]
  \centering
  \fig{Silstrup-Bromide-horizontal-2000}
  
  \caption{Silstrup total horizontal bromide flow between 2000-5-1 and
    2001-5-1.  The flow is shown on the x-axis (positive away from
    drain) as a function of depth shown on the y-axis.  The graph
    labels are the distance from drain in centimeters.}
  \label{fig:Silstrup-Bromide-2000-horizontal}
\end{figure}\FloatBarrier

\begin{figure}[htbp]
  \centering
  \fig{Silstrup-Bromide-2000}
  
  \caption{Silstrup total vertical bromide flow between 2000-5-1 and
    2001-5-1.  The flow is shown on the y-axis (positive up) as a
    function of distance from drain shown on the x-axis.  The graph
    labels are depths in centimeters above surface.}
  \label{fig:Silstrup-Bromide-2000-vertical}
\end{figure}\FloatBarrier

\begin{figure}[htbp]
  \centering
  \fig{Silstrup-Bromide-biopore-2000}
  
  \caption{Silstrup total biopore bromide flow between 2000-5-1 and
    2001-5-1.  The flow is shown on the y-axis (positive up) as a
    function of distance from drain shown on the x-axis.  The graph
    labels are depths in centimeters above surface.}
  \label{fig:Silstrup-Bromide-biopore-2000}
\end{figure}\FloatBarrier

\begin{figure}[htbp]
  \centering
  \fig{Silstrup-Bromide-horizontal-2001}
  
  \caption{Silstrup total horizontal bromide flow between 2001-5-1 and
    2002-3-1.  The flow is shown on the x-axis (positive away from
    drain) as a function of depth shown on the y-axis.  The graph
    labels are the distance from drain in centimeters.}
  \label{fig:Silstrup-Bromide-2001-horizontal}
\end{figure}\FloatBarrier

\begin{figure}[htbp]
  \centering
  \fig{Silstrup-Bromide-2001}
  
  \caption{Silstrup total vertical bromide flow between 2001-5-1 and
    2002-3-1.  The flow is shown on the y-axis (positive up) as a
    function of distance from drain shown on the x-axis.  The graph
    labels are depths in centimeters above surface.}
  \label{fig:Silstrup-Bromide-2001-vertical}
\end{figure}\FloatBarrier

\begin{figure}[htbp]
  \centering
  \fig{Silstrup-Bromide-biopore-2001}
  
  \caption{Silstrup total biopore bromide flow between 2001-5-1 and
    2002-3-1.  The flow is shown on the y-axis (positive up) as a
    function of distance from drain shown on the x-axis.  The graph
    labels are depths in centimeters above surface.}
  \label{fig:Silstrup-Bromide-biopore-2001}
\end{figure}\FloatBarrier

\begin{figure}[htbp]\centering
  \begin{tabular}{ccc}
    \figsilstrupl{Silstrup-M-Metamitron-2000-5} & 
    \figsilstrup{Silstrup-M-Metamitron-2000-6} & 
    \figsilstrup{Silstrup-M-Metamitron-2000-7} \\
    \figsilstrupl{Silstrup-M-Metamitron-2000-8} & 
    \figsilstrup{Silstrup-M-Metamitron-2000-9} & 
    \figsilstrup{Silstrup-M-Metamitron-2000-10} \\
    \figsilstrupl{Silstrup-M-Metamitron-2000-11} & 
    \figsilstrup{Silstrup-M-Metamitron-2000-12} & 
    \figsilstrup{Silstrup-M-Metamitron-2001-1} \\
    \figsilstrupl{Silstrup-M-Metamitron-2001-2} & 
    \figsilstrup{Silstrup-M-Metamitron-2001-3} & 
    \figsilstrup{Silstrup-M-Metamitron-2001-4}
  \end{tabular}
  
  \caption{Silstrup metamitron soil content at the end of each month
    since first application of bromide.  The y-axis denotes depth, the
    x-axis distance from drain.  There are tick marks for every
    meter. The color scale is white<10 pg/l, yellow=1 ng/l, orange=0.1
    $\mu$g/l, red=10 $\mu$g/l, and black>1 mg/l}
\label{fig:Silstrup-M-Metamitron-2000}
\end{figure}\FloatBarrier

\begin{figure}[htbp]\centering
  \begin{tabular}{ccc}
    \figsilstrupl{Silstrup-C-Metamitron-2000-5} & 
    \figsilstrup{Silstrup-C-Metamitron-2000-6} & 
    \figsilstrup{Silstrup-C-Metamitron-2000-7} \\
    \figsilstrupl{Silstrup-C-Metamitron-2000-8} & 
    \figsilstrup{Silstrup-C-Metamitron-2000-9} & 
    \figsilstrup{Silstrup-C-Metamitron-2000-10} \\
    \figsilstrupl{Silstrup-C-Metamitron-2000-11} & 
    \figsilstrup{Silstrup-C-Metamitron-2000-12} & 
    \figsilstrup{Silstrup-C-Metamitron-2001-1} \\
    \figsilstrupl{Silstrup-C-Metamitron-2001-2} & 
    \figsilstrup{Silstrup-C-Metamitron-2001-3} & 
    \figsilstrup{Silstrup-C-Metamitron-2001-4}
  \end{tabular}
  
  \caption{Silstrup metamitron soil water concentration at the end of
    each month since first application of bromide.  The y-axis denotes
    depth, the x-axis distance from drain.  There are tick marks for
    every meter. The color scale is white<10 pg/l, yellow=1 ng/l, orange=0.1
    $\mu$g/l, red=10 $\mu$g/l, and black>1 mg/l}
\label{fig:Silstrup-C-Metamitron-2000}
\end{figure}\FloatBarrier

\begin{figure}[htbp]
  \centering
  \fig{Silstrup-Metamitron-horizontal-2000}
  
  \caption{Silstrup total horizontal metamitron flow between 2000-5-1 and
    2001-5-1.  The flow is shown on the x-axis (positive away from
    drain) as a function of depth shown on the y-axis.  The graph
    labels are the distance from drain in centimeters.}
  \label{fig:Silstrup-Metamitron-2000-horizontal}
\end{figure}\FloatBarrier

\begin{figure}[htbp]
  \centering
  \fig{Silstrup-Metamitron-2000}
  
  \caption{Silstrup total vertical metamitron flow between 2000-5-1 and
    2001-5-1.  The flow is shown on the y-axis (positive up) as a
    function of distance from drain shown on the x-axis.  The graph
    labels are depths in centimeters above surface.}
  \label{fig:Silstrup-Metamitron-2000-vertical}
\end{figure}\FloatBarrier

\begin{figure}[htbp]
  \centering
  \fig{Silstrup-Metamitron-biopore-2000}
  
  \caption{Silstrup total biopore metamitron flow between 2000-5-1 and
    2001-5-1.  The flow is shown on the y-axis (positive up) as a
    function of distance from drain shown on the x-axis.  The graph
    labels are depths in centimeters above surface.}
  \label{fig:Silstrup-Metamitron-biopore-2000}
\end{figure}\FloatBarrier

\begin{figure}[htbp]\centering
  \begin{tabular}{ccc}
    \figsilstrupl{Silstrup-M-Glyphosate-2001-5} & 
    \figsilstrup{Silstrup-M-Glyphosate-2001-6} & 
    \figsilstrup{Silstrup-M-Glyphosate-2001-7} \\
    \figsilstrupl{Silstrup-M-Glyphosate-2001-8} & 
    \figsilstrup{Silstrup-M-Glyphosate-2001-9} & 
    \figsilstrup{Silstrup-M-Glyphosate-2001-10} \\
    \figsilstrupl{Silstrup-M-Glyphosate-2001-11} & 
    \figsilstrup{Silstrup-M-Glyphosate-2001-12} & 
    \figsilstrup{Silstrup-M-Glyphosate-2002-1} \\
    \figsilstrupl{Silstrup-M-Glyphosate-2002-2} & & 
  \end{tabular}
  
  \caption{Silstrup glyphosate soil content at the end of each month
    since one year after the first application of bromide.  The y-axis
    denotes depth, the x-axis distance from drain.  There are tick
    marks for every meter. The color scale is white<10 pg/l, yellow=1
    ng/l, orange=0.1 $\mu$g/l, red=10 $\mu$g/l, and black>1 mg/l}
\label{fig:Silstrup-M-Glyphosate-2001}
\end{figure}\FloatBarrier

\begin{figure}[htbp]\centering
  \begin{tabular}{ccc}
    \figsilstrupl{Silstrup-C-Glyphosate-2001-5} & 
    \figsilstrup{Silstrup-C-Glyphosate-2001-6} & 
    \figsilstrup{Silstrup-C-Glyphosate-2001-7} \\
    \figsilstrupl{Silstrup-C-Glyphosate-2001-8} & 
    \figsilstrup{Silstrup-C-Glyphosate-2001-9} & 
    \figsilstrup{Silstrup-C-Glyphosate-2001-10} \\
    \figsilstrupl{Silstrup-C-Glyphosate-2001-11} & 
    \figsilstrup{Silstrup-C-Glyphosate-2001-12} & 
    \figsilstrup{Silstrup-C-Glyphosate-2002-1} \\
    \figsilstrupl{Silstrup-C-Glyphosate-2002-2} &  & 
  \end{tabular}
  
  \caption{Silstrup glyphosate soil water concentration at the end of
    each month since one year after first application of bromide.  The
    y-axis denotes depth, the x-axis distance from drain.  There are
    tick marks for every meter. The color scale is white<10 pg/l,
    yellow=1 ng/l, orange=0.1 $\mu$g/l, red=10 $\mu$g/l, and black>1
    mg/l}
\label{fig:Silstrup-C-Glyphosate-2001}
\end{figure}\FloatBarrier

\begin{figure}[htbp]
  \centering
  \fig{Silstrup-Glyphosate-horizontal-2001}
  
  \caption{Silstrup total horizontal glyphosate flow between 2001-5-1 and
    2002-3-1.  The flow is shown on the x-axis (positive away from
    drain) as a function of depth shown on the y-axis.  The graph
    labels are the distance from drain in centimeters.}
  \label{fig:Silstrup-Glyphosate-2001-horizontal}
\end{figure}\FloatBarrier

\begin{figure}[htbp]
  \centering
  \fig{Silstrup-Glyphosate-2001}
  
  \caption{Silstrup total vertical glyphosate flow between 2001-5-1 and
    2002-3-1.  The flow is shown on the y-axis (positive up) as a
    function of distance from drain shown on the x-axis.  The graph
    labels are depths in centimeters above surface.}
  \label{fig:Silstrup-Glyphosate-2001-vertical}
\end{figure}\FloatBarrier

\begin{figure}[htbp]
  \centering
  \fig{Silstrup-Glyphosate-biopore-2001}
  
  \caption{Silstrup total biopore glyphosate flow between 2001-5-1 and
    2002-3-1.  The flow is shown on the y-axis (positive up) as a
    function of distance from drain shown on the x-axis.  The graph
    labels are depths in centimeters above surface.}
  \label{fig:Silstrup-Glyphosate-biopore-2001}
\end{figure}\FloatBarrier
