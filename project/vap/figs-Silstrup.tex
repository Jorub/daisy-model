\subsection*{Silstrup}

\begin{figure}[htbp]
  \begin{center}
    \includegraphics{fig/Silstrup-weather}
  \end{center}
  \caption{Accumulated precipitation and hourly values for temperature
    measured at Silstrup station.  Calculated potentiel and simulated
    actual evapotranspiration are also shown.}
  \label{fig:Silstrup-weather}
\end{figure}\FloatBarrier

\begin{figure}[htbp] 
  \includegraphics{fig/Silstrup-Ap-Theta}\includegraphics{fig/Silstrup-Ap-K}\\
  \includegraphics{fig/Silstrup-B-Theta}\includegraphics{fig/Silstrup-B-K}\\
  \includegraphics{fig/Silstrup-C-Theta}\includegraphics{fig/Silstrup-C-K}
  \caption{Silstrup soil hydraulic properties.  MACRO denotes the
    original parametrization, Daisy the modified parametrization, and
    HYPRES refers to parameters estimated according to
    \citet{hypres}.}
  \label{fig:Silstrup-hor}
\end{figure}\FloatBarrier

\begin{figure}[htbp]
  \begin{center}
    \includegraphics{fig/Silstrup-gw}
  \end{center}
  \caption{Silstrup groundwater table.  Automatic daily measurements
    are from P3.  Simulated low value is calculated from pressure in
    lowest unsatured numeric cell, typically located near drain.
    Simulated high value is calculated from pressure in highest
    saturated cell, typically farthest from the drain.}
  \label{fig:Silstrup-gw}
\end{figure}\FloatBarrier

\begin{figure}[htbp]
  \begin{center}
    \includegraphics[trim=0mm 5mm 0mm 0mm,clip]{fig/Silstrup-theta-SW025cm}\\
    \includegraphics[trim=0mm 5mm 0mm 0mm,clip]{fig/Silstrup-theta-SW060cm}\\
    \includegraphics{fig/Silstrup-theta-SW110cm}
  \end{center}
  \caption{Silstrup soil water content for measurement point S1.}
  \label{fig:Silstrup-theta}
\end{figure}\FloatBarrier

\begin{figure}[htbp]
  \begin{center}
    \includegraphics[trim=0mm 5mm 0mm 0mm,clip]{fig/Silstrup-sc-bromide}\\
    \includegraphics{fig/Silstrup-Bromide-horizontal}
  \end{center}
  \caption{Silstrup soil bromide content in 1.0 m depth (top) and 3.5
    m depth (bottom).  Sim (avg) is the average simulated
    concentration, Sim (fast) is the simulated concentration in the
    large (fast) pores.  S1 and S2 are suction cup measurements.
    H$n$.$m$ refer to measured values in different sections of
    horizontal filters.}
  \label{fig:Silstrup-bromide}
\end{figure}\FloatBarrier

\begin{figure}[htbp]
  \begin{center}
    \includegraphics{fig/Silstrup-horizontal}
  \end{center}
  \caption{Silstrup pesticide concentration in soil water at 3.5 meters depth.}
  \label{fig:Silstrup-horizontal}
\end{figure}\FloatBarrier

\begin{figure}[htbp]
  \begin{center}
    \includegraphics[trim=0mm 5mm 0mm 0mm,clip]{fig/Silstrup-leak150bromide}\\
    \includegraphics[trim=0mm 5mm 0mm 0mm,clip]{fig/Silstrup-leak150}\\
    \includegraphics{fig/Silstrup-leak150acc}
  \end{center}
  \caption{Silstrup simuleret leaching at 1.5 meter, 30 cm under bioporers.}
  \label{fig:Silstrup-leak150}
\end{figure}\FloatBarrier

\begin{figure}[htbp]
  \begin{center}
    \includegraphics[trim=0mm 5mm 0mm 0mm,clip]{fig/Silstrup-drain}\\
    \includegraphics{fig/Silstrup-drain-acc}
  \end{center}
  \caption{Silstrup drain flow, daily values and accumulated.}
  \label{fig:Silstrup-drain}
\end{figure}\FloatBarrier

\begin{figure}[htbp]
  \begin{center}
    \includegraphics[trim=0mm 5mm 0mm 0mm,clip]{fig/Silstrup-Bromide-weekly}\\
    \includegraphics{fig/Silstrup-Metamitron-weekly}
  \end{center}
  \caption{Weekly drain flow of bromide and metamitron.}
  \label{fig:Silstrup-weekly}
\end{figure}\FloatBarrier

\begin{figure}[htbp]
  \begin{center}
    \includegraphics[trim=0mm 5mm 0mm 0mm,clip]{fig/Silstrup-Dimethoate-weekly}\\
    \includegraphics[trim=0mm 5mm 0mm 0mm,clip]{fig/Silstrup-Fenpropimorph-weekly}\\
    \includegraphics{fig/Silstrup-Glyphosate-weekly}
  \end{center}
  \caption{Weekly drain flow of selected pesticides.}
  \label{fig:Silstrup-weekly2}
\end{figure}\FloatBarrier

\begin{figure}[htbp]
  \begin{center}
    \includegraphics[trim=0mm 5mm 0mm 0mm,clip]{fig/Silstrup-Bromide-acc}\\
    \includegraphics{fig/Silstrup-Metamitron-acc}
  \end{center}
  \caption{Accumulated drain flow of bromide and metamitron.}
  \label{fig:Silstrup-bromide-acc}
\end{figure}\FloatBarrier

\begin{figure}[htbp]
  \begin{center}
    \includegraphics[trim=0mm 5mm 0mm 0mm,clip]{fig/Silstrup-Dimethoate-acc}\\
    \includegraphics[trim=0mm 5mm 0mm 0mm,clip]{fig/Silstrup-Fenpropimorph-acc}\\
    \includegraphics{fig/Silstrup-Glyphosate-acc}
  \end{center}
  \caption{Accumulated drain flow of selected pesticides.}
  \label{fig:Silstrup-acc}
\end{figure}\FloatBarrier

\begin{figure}[htbp]
  \begin{center}
    \includegraphics{fig/Silstrup-colloid}
  \end{center}
  \caption{Colloids in drain water.}
  \label{fig:Silstrup-colloids}
\end{figure}\FloatBarrier

\begin{figure}[htbp]
  \begin{center}
    \includegraphics[trim=0mm 5mm 0mm 0mm,clip]{fig/Silstrup-biopore}\\
    \includegraphics{fig/Silstrup-biopore-acc}\\
  \end{center}
  \caption{Biopore activity in different soil layers.  The layers are
    ponded water, soil surface (top 3 cm), the rest of the plowing layer,
    the plow pan, and the the B horizon below plow pan down to 50 cm.}
  \label{fig:Silstrup-biopore}
\end{figure}\FloatBarrier

\begin{figure}[htbp]
  \begin{center}
    \includegraphics[trim=0mm 5mm 0mm 0mm,clip]{fig/Silstrup-biopore-drain}\\
    \includegraphics{fig/Silstrup-biopore-drain-acc}
  \end{center}
  \caption{Drain contribution through biopores from different soil
    layers.  The layers are ponded water, soil surface (top 3 cm), the
    rest of the plowing layer, the plow pan, and the the B horizon
    below plow pan down to 50 cm.}
  \label{fig:Silstrup-biopore-drain}
\end{figure}\FloatBarrier

\begin{figure}[htbp]
  \begin{center}
    \includegraphics[trim=0mm 5mm 0mm 0mm,clip]{fig/Silstrup-weather-glyphosate}\\
    \includegraphics[trim=0mm 5mm 0mm 0mm,clip]{fig/Silstrup-water-glyphosate}\\
    \includegraphics{fig/Silstrup-first-glyphosate}
  \end{center}
  \caption{Silstrup surface water and glyphosate in the first week
    after application.  Top graph shows fluxes affecting surface
    water.  Middle graph shows water storage on surface, as well as
    the water holding capcaity of the litter pack.  Bottom graph track
    the fate of glyphosate on the surface.}
  \label{fig:Silstrup-weather-glyphosate}
\end{figure}\FloatBarrier

