\documentclass[a4paper]{report}

%%\usepackage[left=1cm,top=2cm,right=1cm]{geometry}
\usepackage[top=3cm,bottom=2cm]{geometry}
\usepackage[latin1]{inputenc}
\usepackage[T1]{fontenc}
\usepackage[danish,english]{babel}
\usepackage{natbib}
\bibliographystyle{apalike}
\usepackage{graphicx}
\usepackage{hyperref}
\usepackage{fancyhdr}
\usepackage{placeins}
\pagestyle{fancy}
\lhead{\today}

\newcommand{\koc}{$\mbox{K}_{\mbox{\textsc{oc}}}$}
\newcommand{\kclay}{$\mbox{K}_{\mbox{clay}}$}
\newcommand{\kd}{$\mbox{K}_{\mbox{d}}$}

\newcommand{\focus}{\textsc{focus}}
\newcommand{\hypres}{\textsc{hypres}}
\newcommand{\Hypres}{\textsc{Hypres}}
\newcommand{\macro}{\textsc{macro}}
\newcommand{\Macro}{\textsc{Macro}}
\newcommand{\figl}{\hspace*{-2cm}}
\newcommand{\figright}[1]{\includegraphics{fig/#1}}
\newcommand{\fig}[1]{\figl\figright{#1}}
\newcommand{\figtop}[1]{\figl\includegraphics[trim=0mm 5mm 0mm 0mm,clip]{fig/#1}}

\newcommand{\figctop}[1]{\hspace*{-1cm}\figright{#1}} 
\newcommand{\figc}[1]{\vspace*{-1.5cm}\figctop{#1}}

\newcommand{\MyID}{}

\begin{document}

\chapter*{Daisy 2D simulation of R�rrendeg�rd}

Part of project
\begin{otherlanguage}{danish}
  \begin{it}
    Flerdimensional modelling of vandstr�mning og stoftransport i de
    �verste 1-2 m af jorden i systemer med markdr�n
  \end{it}
\end{otherlanguage}
for the Danish Environmental Protection Agency.
\vspace{1cm}

\begin{bf}
  \begin{large}
    \noindent
    S�ren Hansen \texttt{$<$sha@life.ku.dk$>$}\\
    Per Abrahamsen \texttt{$<$abraham@dina.kvl.dk$>$}\\
    Carsten Pedersen \texttt{$<$cpe@life.ku.dk$>$}\\
    Marie Habekost Nielsen \texttt{$<$maha@life.ku.dk$>$}\\
    Mikkel Mollerup \texttt{$<$mmo@geus.dk$>$}\\
    \\
    \today{}\\
  \end{large}
\end{bf}
\vfill\noindent
University of Copenhagen\\
Department of Basic Sciences and Environment\\
Environmental Chemistry and Physics\\
Thorvaldsensvej 40\\
DK-1871 Frederiksberg C\\
Tel: \texttt{$+$45 353 32300}\\
Fax: \texttt{$+$45 353 32398}

\tableofcontents

\chapter{Introduction}

The R�rrendeg�rd site is part of the Copenhagen University
experimental station in T�strup.  It was selected for for the present
project mainly because soil particles in drain water were measured,
and soil particles can be an important transport mechanism for
strongly sorbing pesticides, e.g. \citet{}.   

The Agrovand project measured drain water together  particles 

The results have been partly documented in \citet{Petersen200181}

because of the high resolution flow proportional drain data collected
as part of the Agrovand project in the four drain season between
between 1998 and 2002, which included soil particles, a likely
transport path for strongly sorbing pesticides.

\begin{verbatim}

The data from the R�rrendeg�rd site is mostly taken
The Agrovand project 
- KU experimental field R�rrendeg�rd
- Effect of tillage on drain water and transport to drains
- Some results published in

Old data
- Four seasons
- Four tillage regimenes
- Particles, TDR, Piezometers
- Bromide, Pesticides

New data
- Biopores

Our interest

Limitations
\end{verbatim}

\chapter{Setup}

Process \citet{tilde-agrovand,nanna-agrovand}, mmo, \citet{vap2d}, pa

\section{Weather}

\section{Management}

\section{Pesticide and bromide properties}

\section{Soil}

\subsection{Matrix}

Figure~\ref{fig:Rorrende-hor}.

\begin{figure}[htbp] 
  \fig{Rorrende-Ap-Theta}\figright{Rorrende-Ap-K}\\
  \fig{Rorrende-Bt-Theta}\figright{Rorrende-Bt-K}\\
  \fig{Rorrende-C-Theta}\figright{Rorrende-C-K}\\
  \fig{Rorrende-DC-Theta}\figright{Rorrende-DC-K}
  \caption{R{\o}rrende soil hydraulic properties.  \Hypres{} refers to
    parameters estimated according to \citet{hypres}, Daisy to the
    final parametrization (ignoring anisotropy and biopores), and
    Surface and plow pan to the conditions at the top of the A and Bt
    horizons.  DC is the drain canyon.}
  \label{fig:Rorrende-hor}
\end{figure}

\subsection{Fast and slow water}

The division between fast and slow water for the Ap horizon were
mainly calibrated from the measurements shown on the top graph on
figure~\ref{fig:bromide-acc}.  The simulated dynamics shown on
figure~\ref{fig:bromide} were used as a help.  The two figures are
explined in section~\ref{sec:soil-bromide}.

\subsection{Biopores}

\subsection{Groundwater table and drain pipes}

\subsection{Organic matter and nitrogen}

\chapter{Results}

The simulation results are presented together with measured data in
figures~\ref{fig:first} to~\ref{fig:last}, found at the end of the
report.  The figures have a high information density, and have
therefore been allowed to fill most of the page.  Each figure contains
multiple graphs, all of which share the same x-axis.  This structure
is intended to facilitate comparison.

\section{Soil water}

\section{Soil bromide}
\label{sec:soil-bromide}

Same pattern in all treatments.

figure~\ref{fig:bromide} and figure~\ref{fig:bromide-acc}

\section{Drains}

The full drain seasons are depicted on
figure~\ref{fig:season9899},~\ref{fig:season9900},
and~\label{fig:season0001}, while
figure~\ref{fig:season9899zoom},~\ref{fig:season9900zoom},
and~\label{fig:season0001zoom} focus on a single event within each
drain season.

\chapter{Discussion}

\section{Colloid generation}

The most signficant part of the dataset is the colloid leaching for
different soil treatment regimes.  We have only looked at the
conventional tillage regime, and used that for calibrating the colloid
generation model described in \citet{macro-colloid}.  By using data
from the three other regimes, it should be possible to improve the
model to takes into account the timing of tillage operations, and help
predict the possible effect of low tillage regimes on pesticide
leaching.

\addcontentsline{toc}{chapter}{\numberline{}References}
\bibliography{../../txt/daisy}

\appendix{}

%% Drain figures

\newgeometry{left=1cm,top=1cm,right=1cm,bottom=1cm,nohead,nofoot}
\pagestyle{empty}
\begin{figure}[htbp]
  \begin{center}
    \figc{weather} \\
    \figc{theta4cm} \\
    \figc{theta8cm} \\
    \figc{theta12cm} \\
    \figc{theta16cm} \\
    \figc{theta20cm} \\
    \figc{theta24cm} \\
    \figc{theta36cm} \\
    \figc{theta60cm}
  \end{center}
  \caption{\MyID{}TDR measurements.}
  \label{fig:tdr}
  \label{fig:first}
\end{figure}

\begin{figure}[htbp]
  \begin{center}
    \figctop{weather_short} \\
    \figc{theta_short4cm} \\
    \figc{theta_short8cm} \\
    \figc{theta_short12cm} \\
    \figc{theta_short16cm} \\
    \figc{theta_short20cm} \\
    \figc{theta_short24cm} \\
    \figc{theta_short36cm} \\
    \figc{theta_short60cm}
  \end{center}
  \caption{\MyID{}Early TDR measurements.}
  \label{fig:tdr-zoom}
\end{figure}

\begin{figure}[htbp]
  \begin{center}
    \figctop{weather_brominf} \\
    \figc{infiltration}\\
    \figc{pondingdepth}\\
    \figc{brom-input} \\
    \figc{brom-0-25-output} \\
    \figc{brom-25-50-output} \\
    \figc{brom-50-75-output} \\
    \figc{brom-75-100-output}
  \end{center}
  \caption{\MyID{}Bromide dynamics.}
  \label{fig:bromide}
\end{figure}

\begin{figure}[htbp]
  \begin{center}
    \figctop{brom-total} \\
    \figc{brom-primary} \\
    \figc{brom-secondary} \\
    \figc{brom-input-acc} \\
    \figc{brom-0-25-acc} \\
    \figc{brom-25-50-acc} \\
    \figc{brom-50-75-acc} \\
    \figc{brom-75-100-acc}
  \end{center}
  \caption{\MyID{}Accumulated bromide.}
  \label{fig:bromide-acc}
\end{figure}

\begin{figure}[htbp]
  \begin{center}
    \figctop{weather-98-99} \\
    \figc{drainflow-98-99} \\
    \figc{drainflowacc-98-99} \\
    \figc{particles-98-99} \\
    \figc{particlesacc-98-99} \\
    \figc{bromide-98-99} \\
    \figc{brommass-98-99}
  \end{center}
  \caption{\MyID{}Drain season 1998 -- 1999.}
  \label{fig:season9899}
\end{figure}

\begin{figure}[htbp]
  \begin{center}
    \figctop{weather-98-99-zoom} \\
    \figc{drainflow-98-99-zoom} \\
    \figc{drainflowacc-98-99-zoom} \\
    \figc{particles-98-99-zoom} \\
    \figc{particlesacc-98-99-zoom} \\
    \figc{bromide-98-99-zoom} \\
    \figc{brommass-98-99-zoom}
  \end{center}
  \caption{\MyID{}Drain season 1998 --- 1999, single event.}
  \label{fig:season9899zoom}
\end{figure}

\begin{figure}[htbp]
  \begin{center}
    \figctop{weather-99-00} \\
    \figc{drainflow-99-00} \\
    \figc{drainflowacc-99-00} \\
    \figc{particles-99-00} \\
    \figc{particlesacc-99-00} \\
    \figc{bromide-99-00} \\
    \figc{brommass-99-00} \\
    \figc{pendconc-99-00} \\
    \figc{pendmass-99-00}
  \end{center}
  \caption{\MyID{}Drain season 1999 --- 2000.}
  \label{fig:season9900}
\end{figure}

\begin{figure}[htbp]
  \begin{center}
    \figctop{weather-99-00-zoom} \\
    \figc{drainflow-99-00-zoom} \\
    \figc{drainflowacc-99-00-zoom} \\
    \figc{particles-99-00-zoom} \\
    \figc{particlesacc-99-00-zoom} \\
    \figc{bromide-99-00-zoom} \\
    \figc{brommass-99-00-zoom} \\
    \figc{pendconc-99-00-zoom} \\
    \figc{pendmass-99-00-zoom}
  \end{center}
  \caption{\MyID{}Drain season 1999 --- 2000, single event.}
  \label{fig:season9900zoom}
\end{figure}

\begin{figure}[htbp]
  \begin{center}
    \figctop{weather-00-01} \\
    \figc{drainflow-00-01} \\
    \figc{drainflowacc-00-01} \\
    \figc{particles-00-01} \\
    \figc{particlesacc-00-01} \\
    \figc{ioxconc-00-01} \\
    \figc{ioxmass-00-01} \\
    \figc{pendconc-00-01} \\
    \figc{pendmass-00-01}
  \end{center}
  \caption{\MyID{}Drain season 2000 --- 2001.}
  \label{fig:season0001}
\end{figure}

\begin{figure}[htbp]
  \begin{center}
    \figctop{weather-00-01-zoom} \\
    \figc{drainflow-00-01-zoom} \\
    \figc{drainflowacc-00-01-zoom} \\
    \figc{particles-00-01-zoom} \\
    \figc{particlesacc-00-01-zoom} \\
    \figc{ioxconc-00-01-zoom} \\
    \figc{ioxmass-00-01-zoom} \\
    \figc{pendconc-00-01-zoom} \\
    \figc{pendmass-00-01-zoom}
  \end{center}
  \caption{\MyID{}Drain season 2000 --- 2001, single event.}
  \label{fig:season0001zoom}
  \label{fig:last}
\end{figure}

%%% Local Variables: 
%%% mode: latex
%%% TeX-master: nil
%%% End: 


\end{document}

%%% Local Variables: 
%%% mode: latex
%%% TeX-master: t
%%% End: 
